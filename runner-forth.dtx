% \iffalse
%% File: runner-forth.dtx Copyright (C) 2013 Bruno Le Floch
%%
%% It may be distributed and/or modified under the conditions of the
%% LaTeX Project Public License (LPPL), either version 1.3c of this
%% license or (at your option) any later version.  The latest version
%% of this license is in the file
%%
%%    http://www.latex-project.org/lppl.txt
%%
%% -----------------------------------------------------------------------
% \fi
%
% \begin{runner-documentation}
%
% \newcommand{\kw}[1]{\texttt{#1}}
%
% \section{The \plang{forth} language}
%
% This is not the place for an introduction to Forth, many can be found
% online.  ^^A todo: add a bunch of refs
%
% For instance, one can define the factorial function as follows.
% \begin{verbatim}
%   \runner{forth}[output = \result]|
%     : FACTORIAL ( +n1 -- +n2)
%     DUP 2 < IF DROP 1 ELSE ( 0! = 1! = 1)
%     DUP 1- RECURSE * THEN ; ( n1! = n1 * [n1 - 1]!)
%   ||
%   CR 0 FACTORIAL . ( prints 1)
%   CR 3 FACTORIAL . ( prints 6)
%   CR 24 FACTORIAL . ( prints 12582912)|
%   \typeout{\result}
% \end{verbatim}
% This types three lines to the terminal, containing \(1\), \(6\) and
% \(12582912\).  The last number is incorrect, because the result
% overflows the size of a single-cell unsigned integer.  We can easily
% check that it is the correct result modulo \(2^{24}\).
%
% The aim is to make of this \plang{forth} interpreter a ``Standard
% System'' as defined by the ANS Forth 1994 specification.  This is not
% yet the case.  The following words are not implemented:
% \begin{itemize}
% \item definitions |CREATE|, |DOES>|, |>BODY|,
% \item control-flow |BEGIN|, |UNTIL|, |WHILE|, |REPEAT|
% \item do-loop |+LOOP|,
% \item parse/input |>IN|, |ACCEPT|, |SOURCE|
% \item ending |ABORT|, |ABORT"|, |QUIT|.
% \item number conversion |<#|, |#|, |#>|, |#S|, |>NUMBER|, |HOLD|,
%   |SIGN|
% \item int |FM/MOD|, |M*|, |S>D|, |SM/REM|, |U.|, |U<|, |UM*|, |UM/MOD|
% \item bitwise |AND|, |INVERT|, |LSHIFT|, |OR|, |RSHIFT|, |XOR|
% \item misc |EVALUATE|, |EXIT|, |FIND|, |KEY|, |MOVE|, |STATE|, |[|,
%   |]|
% \end{itemize}
%
% The following environment variables are not implemented:
% |/COUNTED-STRING|, |/HOLD|.
%
% Implementation specificities are as follows.
% \begin{itemize}
% \item All addresses are aligned and are character-aligned.
% \item Behaviour of |EMIT| for non-graphic character: ?
% \item Character editing of |ACCEPT|: ?
% \item Character set: ?
% \item Character-set-extensions matching characteristics: ?
% \item Conditions under which control characters match a space
%   delimiter: ?
% \item Format of the control-flow stack: ?
% \item Conditions under which control characters match a space delimiter: ?
% \item Format of the control-flow stack: ?
% \item Conversion of digits larger than thirty-five: this never arises,
%   as the base used in conversion is automatically truncated to the
%   interval \([2, 36]\).
% \item Display after input terminates in |ACCEPT|: ?
% \item Exception abort sequence (as in |ABORT"|): ?
% \item Input line terminator: ?  User input device: ?
% \item Maximum size of a counted string, in chars: ?
% \item Maximum size of a parsed string: ?
% \item Maximum size of a definition name, in chars: ?
% \item Maximum string length for |ENVIRONMENT?|, in chars: ?
% \item Method of selecting a user input device: ?
% \item Method of selecting a user output device: ?
% \item Methods of dictionary compilation: ?
% \item Number of bits in one address unit: \(24\).
% \item Number representation and arithmetic: internally represented as
%   \(24\)-bit unsigned integers (well, further down as \TeX{}
%   dimensions in a \tn{fontdimen} array).
% \item Integers \([-2^{23}, 2^{23} - 1]\), non-negative integers \([0,
%   2^{23} - 1]\), unsigned integers \([0, 2^{24} - 1]\), double
%   integers \([-2^{47}, 2^{47} - 1]\), non-negative double integers
%   \([0, 2^{47} - 1]\), unsigned double integers \([0, 2^{48} - 1]\).
% \item Read-only data-space regions: ?
% \item Size of buffer at |WORD|: ?
% \item One cell is one address unit.
% \item One character is one address unit.
% \item Size of the keyboard terminal input buffer: fixed upon starting
%   the Forth interpreter, at least \(128\).
% \item Size of the pictured numeric output string buffer: ?
% \item Size of the scratch area whose address is returned by |PAD|:
%   fixed upon starting the Forth interpreter, at least \(128\).
% \item Case-sensitivity: ?
% \item Prompt (see |QUIT|): ?
% \item Division rounding: symmetric (the quotient is rounded towards
%   zero, the remainder has the sign of the numerator).
% \item Values of |STATE| when true: ?
% \item Values returned after arithmetic overflow: value modulo
%   \(2^{24}\).
% \item Whether the current definition can be found after |DOES>|: ?
% \end{itemize}
% ^^A todo: ensure that input buffer and pad are both at least 128 cells wide.
%
% Here is a list of data types and their sizes in this implementation.
% \begin{center}
%   \begin{tabular}{ccc}
%     \toprule
%     Symbol    & Data type  & \# bits \\
%     \midrule
%     flag      & flag (true or false)            & 24 \\
%     char      & character                       & 24 \\
%     n         & signed number                   & 24 \\
%     +n        & non-negative number             & 24 \\
%     u         & unsigned number                 & 24 \\
%     x         & unspecified cell                & 24 \\
%     xt        & execution token                 & 24 \\
%     addr      & address ($=\text{a-addr}=\text{c-addr}$) & 24 \\
%     \midrule
%     d         & double-cell signed number       & 48 \\
%     +d        & double-cell non-negative number & 48 \\
%     ud        & double-cell unsigned number     & 48 \\
%     xd        & unspecified cell pair           & 48 \\
%     \midrule
%     ^^A todo: clarify what those '-sys' types are in this implementation.
%     colon-sys & definition compilation          & dep \\
%     do-sys    & do-loop structures              & dep \\
%     case-sys  & CASE structures                 & dep \\
%     of-sys    & OF structures                   & dep \\
%     orig      & control-flow origins            & dep \\
%     dest      & control-flow destinations       & dep \\
%     loop-sys  & loop-control parameters         & dep \\
%     nest-sys  & definition calls                & dep \\
%     i*x, j*x, k*x (3) & any data type           & \(\geq 0\) \\
%     \bottomrule
%   \end{tabular}
% \end{center}
%
% \end{runner-documentation}
%
% \begin{runner-implementation}
%
% \section{\plang{forth} implementation}
%
% Sections to be re-read:
% \begin{itemize}
% \item 3.1.5. System types
% \item 3.2.1.2. Digit conversion
% \item 3.2.3.3. Return stack
% \item 3.2.4. Operator terminal (decide whether the Forth program can
%   change where it gets its input from)
% \item 3.3.
% \end{itemize}
%
% Maybe relevant text in the 1994 standard.
% \begin{quote}
%   This Standard designates the following words as obsolescent:
%   \begin{itemize}
%   \item 6.2.0060  |#TIB|
%   \item 15.6.2.1580  |FORGET|
%   \item 6.2.2240  |SPAN|
%   \item 6.2.0970  |CONVERT|
%   \item 6.2.2040  |QUERY|
%   \item 6.2.2290  |TIB|
%   \item 6.2.1390  |EXPECT|
%   \end{itemize}
%   [...]\\
%   A system need not provide any standard words for accessing mass
%   storage.  If a system provides any standard word for accessing mass
%   storage, it shall also implement the Block word set.
%   [...]\\
% \end{quote}
%
% The standard says ``char'' is a subtype of ``+n'', so characters are
% only allowed to take values in the range \([0, 2^{23}-1]\)?  Addresses
% are only allowed to take values in the range \([-2^{23}, 2^{23}-1]\)?
%
% Am I allowed to always consider the data-space pointer to be aligned,
% separating cells by \(1\) address unit?
%
% Two useful variants.
%    \begin{macrocode}
\cs_generate_variant:Nn \prop_gput:Nnn { Nx }
\cs_generate_variant:Nn \prop_get:NnNTF { Nx }
%    \end{macrocode}
%
% \subsection{Data structures}
%
% The data necessary to run Forth consists of
% \begin{itemize}
% \item the data stack, manipulated by the program, contains cells,
% \item the control-flow stack, which contains ``-sys'' types, and can
%   be implemented using the data stack,
% \item the return stack, which can be used by the program, but is
%   mostly used by the system, for definitions, do-loops, nesting info,
% \end{itemize}
% as well as
% \begin{itemize}
% \item the name space (mapping words to execution tokens),
% \item the code space (code for the execution tokens),
% \item the data space accessible to programs, which consists in
%   \begin{itemize}
%   \item contiguous regions,
%   \item variables,
%   \item text-literal regions,
%   \item input buffers,
%   \item and other transient regions.
%   \end{itemize}
% \end{itemize}
%
% The name space is implemented as a property list, mapping words to a
% flag in \(\{1,2,3\}\) and an execution token (integer in
% \([0,65535]\)) associated to this word.
%
% The code space is implemented using \tn{toks} registers: execution
% tokens are register numbers.
%
% Everything else is stored as one global array, \cs{g_@@_forth_array},
% of \(24\)-bit unsigned integers.  See \texttt{runner.dtx} for the
% array implementation.
%
% \subsubsection{Defining data space and stacks}
%
% The array \cs{g_@@_forth_array} is split into several regions, in the
% following order (from low-numbered to high-numbered items).  For
% conciseness we refer in this list to integer variables by shorthands;
% \texttt{min} and \texttt{max} variables are global and can only change
% between Forth runs; other integers are local and change during runs.
% \begin{itemize}
% \item The input buffer is from \texttt{input_min} (included) to
%   \texttt{input_max} (excluded).
% \item The pad is from \texttt{pad_min} (included) to \texttt{pad_max}
%   (excluded).
% \item The data space lies from \texttt{data_min} (included) to
%   \texttt{data_max} (excluded), and is split further.
%   \begin{itemize}
%   \item Contiguous regions of the data space, intermixed with
%     variables, lie between \texttt{data_min} (included) and
%     \texttt{data_here} (excluded).
%   \item The transient regions provided by |WORD| or |#>| start at
%     \texttt{data_here} (included) and end at \texttt{data_word}
%     (excluded).
%   \item Text literals are allocated at the end of the data space, and
%     lie between \texttt{data_text} (included) and \texttt{data_max}
%     (excluded).
%   \end{itemize}
% \item The stack, combining the data and control-flow stacks, lies from
%   \texttt{stack_min} to \texttt{stack_max}, and the position where a
%   new value would be pushed is at \texttt{stack_here}.
% \item The return stack lies from \texttt{return_min} to
%   \texttt{return_max}, and the position where a new value would be
%   pushed is at \texttt{return_here}.
% \end{itemize}
%
% In each slot of the array, we store a \(24\) bit unsigned integer,
% \ie, a number in the range \([0, 2^{24}-1]\) (note that we could go up
% to \(26\) bits with no adverse effect).  Address units are \(24\) bits
% wide.  Characters are one address unit wide (note that all Unicode
% code points are less than \(2^{23}\)).  Cells are one character wide.
% We do not use \(8\) bit characters (and addresses) for four reasons:
% only \(256\) cells could be addressed; counted strings would be
% bounded to \(255\) characters; extracting an \(8\)-bit part from a
% \(24\) bit value in \TeX{} is not fast; and we would need to chose an
% encoding for Unicode (this is still needed in \pdfTeX{}, but only at
% the very last step of output).
%
% \begin{variable}{\g_@@_forth_input_min_int, \g_@@_forth_input_max_int}
%   The input buffer ranges from \cs{g_@@_forth_input_min_int}
%   (included) to \cs{g_@@_forth_input_max_int} (excluded).
%    \begin{macrocode}
\int_new:N \g_@@_forth_input_min_int
\int_new:N \g_@@_forth_input_max_int
%    \end{macrocode}
% \end{variable}
%
% \begin{variable}{\g_@@_forth_pad_min_int, \g_@@_forth_pad_max_int}
%   The pad returned by |PAD| is fixed and ranges from
%   \cs{g_@@_forth_pad_min_int} (included) to
%   \cs{g_@@_forth_pad_max_int} (excluded).
%    \begin{macrocode}
\int_new:N \g_@@_forth_pad_min_int
\int_new:N \g_@@_forth_pad_max_int
%    \end{macrocode}
% \end{variable}
%
% \begin{variable}
%   {
%     \g_@@_forth_data_min_int,
%     \l_@@_forth_data_here_int,
%     \l_@@_forth_data_word_int,
%     \l_@@_forth_data_text_int,
%     \g_@@_forth_data_max_int,
%   }
%   Allotted data, and variables, range from
%   \cs{g_@@_forth_data_min_int} (included) to
%   \cs{l_@@_forth_data_here_int} (excluded).  The latter integer is
%   what |HERE| returns.  Transient regions created by |WORD| or |#>|
%   lie between this address (included) and
%   \cs{l_@@_forth_data_word_int} (excluded).  Finally, text strings are
%   stored from \cs{l_@@_forth_data_text_int} (included) to
%   \cs{g_@@_forth_data_max_int} (excluded).
%    \begin{macrocode}
\int_new:N \g_@@_forth_data_min_int
\int_new:N \l_@@_forth_data_here_int
\int_new:N \l_@@_forth_data_word_int
\int_new:N \l_@@_forth_data_text_int
\int_new:N \g_@@_forth_data_max_int
%    \end{macrocode}
% \end{variable}
%
% \begin{variable}
%   {
%     \g_@@_forth_stack_min_int,
%     \l_@@_forth_stack_here_int,
%     \g_@@_forth_stack_max_int
%   }
%   The combination of the data stack and the control flow stack lies in
%   this range, with the bottom item at \cs{g_@@_forth_stack_min_int}
%   and the top item just below \cs{l_@@_forth_stack_here_int}.
%    \begin{macrocode}
\int_new:N \g_@@_forth_stack_min_int
\int_new:N \l_@@_forth_stack_here_int
\int_new:N \g_@@_forth_stack_max_int
%    \end{macrocode}
% \end{variable}
%
% \begin{variable}
%   {
%     \g_@@_forth_return_min_int,
%     \l_@@_forth_return_here_int,
%     \g_@@_forth_return_max_int
%   }
%   The return stack lies there, with the bottom item at
%   \cs{g_@@_forth_return_min_int} and the top of the stack just below
%   \cs{l_@@_forth_return_here_int}.
%    \begin{macrocode}
\int_new:N \g_@@_forth_return_min_int
\int_new:N \l_@@_forth_return_here_int
\int_new:N \g_@@_forth_return_max_int
%    \end{macrocode}
% \end{variable}
%
% \begin{macro}[int]{\@@_forth_init_data:}
%   When starting a Forth run, one must reset the pointers that vary, to
%   the appropriate boundary of their region.
%    \begin{macrocode}
\cs_new_protected_nopar:Npn \@@_forth_init_data:
  {
    \int_set_eq:NN \l_@@_forth_data_here_int
      \g_@@_forth_data_min_int
    \int_set_eq:NN \l_@@_forth_data_word_int
      \g_@@_forth_data_min_int
    \int_set_eq:NN \l_@@_forth_data_text_int
      \g_@@_forth_data_max_int
    \int_set_eq:NN \l_@@_forth_stack_here_int
      \g_@@_forth_stack_min_int
    \int_set_eq:NN \l_@@_forth_return_here_int
      \g_@@_forth_return_min_int
  }
%    \end{macrocode}
% \end{macro}
%
% ^^A todo: make the array size dynamical
%
% \begin{variable}{\g_@@_forth_array}
%   This array contains most of the data needed by Forth.  We also give
%   here initial values for the array parameters, all at once to avoid
%   mistakes.
%    \begin{macrocode}
\int_gset:Nn \g_@@_forth_input_min_int { 0 }
\int_gset:Nn \g_@@_forth_input_max_int { 128 }
\int_gset:Nn \g_@@_forth_pad_min_int { 128 }
\int_gset:Nn \g_@@_forth_pad_max_int { 256 }
\int_gset:Nn \g_@@_forth_data_min_int { 256 }
\int_gset:Nn \g_@@_forth_data_max_int { 65280 }
\int_gset:Nn \g_@@_forth_stack_min_int { 65280 }
\int_gset:Nn \g_@@_forth_stack_max_int { 65408 }
\int_gset:Nn \g_@@_forth_return_min_int { 65408 }
\int_gset:Nn \g_@@_forth_return_max_int { 65536 }
\@@_array_new:Nn \g_@@_forth_array { 65536 }
%    \end{macrocode}
% \end{variable}
%
% \begin{variable}{\c_@@_forth_mod_int}
%   Cells can take up to \(2^{24}\) values.  Arithmetic is often
%   performed modulo that number.
%    \begin{macrocode}
\int_const:Nn \c_@@_forth_mod_int { 16777216 }
%    \end{macrocode}
% \end{variable}
%
% \subsubsection{Manipulating data space}
%
% \begin{macro}[int, EXP]{\@@_forth_data_get:N, \@@_forth_data_get:n}
%   Retrieving unsigned integers from data space.
%    \begin{macrocode}
\cs_new_nopar:Npn \@@_forth_data_get:N
  { \@@_array_item:NN \g_@@_forth_array }
\cs_new_nopar:Npn \@@_forth_data_get:n
  { \@@_array_item:Nn \g_@@_forth_array }
%    \end{macrocode}
% \end{macro}
%
% \begin{macro}[int]{\@@_forth_data_gset:nn}
%   Storing unsigned integers into data space.
%    \begin{macrocode}
\cs_new_protected_nopar:Npn \@@_forth_data_gset:nn
  { \@@_array_gset:Nnn \g_@@_forth_array }
%    \end{macrocode}
% \end{macro}
%
% \begin{macro}[int]{\@@_forth_put_here:n}
%   Put a value at the position given by the \texttt{data_here} integer,
%   then increment that integer.  Check before-hand that we do not reach
%   the region occupied by text literals.
%    \begin{macrocode}
\cs_new_protected:Npn \@@_forth_put_here:n #1
  {
    \int_compare:nNnTF
      \l_@@_forth_data_here_int < \l_@@_forth_data_text_int
      {
        \@@_forth_data_gset:nn { \l_@@_forth_data_here_int } {#1}
        \int_incr:N \l_@@_forth_data_here_int
      }
      { \msg_error:nn { runner/forth } { out-of-memory } }
  }
%    \end{macrocode}
% \end{macro}
%
% \subsubsection{Manipulating the data stack}
%
% \begin{macro}[int]{\@@_forth_pop_int:N}
%   Get data from the stack.  The stack is empty if \texttt{stack_here}
%   is \texttt{stack_min}.
%    \begin{macrocode}
\cs_new_protected:Npn \@@_forth_pop_int:N #1
  {
    \int_compare:nNnTF
      \l_@@_forth_stack_here_int > \g_@@_forth_stack_min_int
      {
        \int_decr:N \l_@@_forth_stack_here_int
        \int_set:Nn #1
          {
            \@@_array_item:NN \g_@@_forth_array
              \l_@@_forth_stack_here_int
          }
      }
      { \msg_error:nn { runner/forth } { empty-stack } }
  }
%    \end{macrocode}
% \end{macro}
%
% ^^A todo: Improve bound-checking.
% ^^A todo: Are the multi-pop useful?  Yes, if single bound checking.
% \begin{macro}[int]
%   {
%     \@@_forth_pop_int:NN,
%     \@@_forth_pop_int:NNN,
%     \@@_forth_pop_int:NNNN
%   }
%   Get multiple integers from stack.  Note that things are popped
%   backwards, so that the last argument gets the top of the stack.
%    \begin{macrocode}
\cs_new_protected:Npn \@@_forth_pop_int:NN #1#2
  {
    \@@_forth_pop_int:N #2
    \@@_forth_pop_int:N #1
  }
\cs_new_protected:Npn \@@_forth_pop_int:NNN #1#2#3
  {
    \@@_forth_pop_int:N #3
    \@@_forth_pop_int:N #2
    \@@_forth_pop_int:N #1
  }
\cs_new_protected:Npn \@@_forth_pop_int:NNNN #1#2#3#4
  {
    \@@_forth_pop_int:N #4
    \@@_forth_pop_int:N #3
    \@@_forth_pop_int:N #2
    \@@_forth_pop_int:N #1
  }
%    \end{macrocode}
% \end{macro}
%
% \begin{macro}[int]{\@@_forth_push:n}
%   Push an unsigned integer onto the stack, at \texttt{stack_here},
%   failing if the stack is full.
%    \begin{macrocode}
\cs_new_protected:Npn \@@_forth_push:n #1
  {
    \int_compare:nNnTF
      \l_@@_forth_stack_here_int < \g_@@_forth_stack_max_int
      {
        \@@_array_gset:Nnn \g_@@_forth_array
          \l_@@_forth_stack_here_int {#1}
        \int_incr:N \l_@@_forth_stack_here_int
      }
      { \msg_error:nn { runner/forth } { out-of-memory } }
  }
%    \end{macrocode}
% \end{macro}
%
% \begin{macro}[int]
%   {
%     \@@_forth_push:nn,
%     \@@_forth_push:nnn,
%     \@@_forth_push:nnnn,
%     \@@_forth_push:nnnnn,
%     \@@_forth_push:nnnnnn,
%   }
%   Push multiple unsigned integers.
%    \begin{macrocode}
\cs_new_protected:Npn \@@_forth_push:nn #1#2
  {
    \@@_forth_push:n {#1}
    \@@_forth_push:n {#2}
  }
\cs_new_protected:Npn \@@_forth_push:nnn #1#2#3
  {
    \@@_forth_push:n {#1}
    \@@_forth_push:n {#2}
    \@@_forth_push:n {#3}
  }
\cs_new_protected:Npn \@@_forth_push:nnnn #1#2#3#4
  {
    \@@_forth_push:n {#1}
    \@@_forth_push:n {#2}
    \@@_forth_push:n {#3}
    \@@_forth_push:n {#4}
  }
\cs_new_protected:Npn \@@_forth_push:nnnnn #1#2#3#4#5
  {
    \@@_forth_push:n {#1}
    \@@_forth_push:n {#2}
    \@@_forth_push:n {#3}
    \@@_forth_push:n {#4}
    \@@_forth_push:n {#5}
  }
\cs_new_protected:Npn \@@_forth_push:nnnnnn #1#2#3#4#5#6
  {
    \@@_forth_push:n {#1}
    \@@_forth_push:n {#2}
    \@@_forth_push:n {#3}
    \@@_forth_push:n {#4}
    \@@_forth_push:n {#5}
    \@@_forth_push:n {#6}
  }
%    \end{macrocode}
% \end{macro}
%
% \begin{macro}[int]{\@@_forth_push_false:, \@@_forth_push_true:}
%   The \texttt{false} value is \(0\), the true value is \(-1\), which
%   we store as a positive integer by shifting by
%   \cs{c_@@_forth_mod_int}.
%    \begin{macrocode}
\cs_new_protected_nopar:Npn \@@_forth_push_false:
  { \@@_forth_push:n { 0 } }
\cs_new_protected_nopar:Npx \@@_forth_push_true:
  {
    \@@_forth_push:n
      { \int_eval:n { \c_@@_forth_mod_int - 1 } }
  }
%    \end{macrocode}
% \end{macro}
%
% \begin{macro}[int]{\@@_forth_signed:N}
%   Turn the unsigned |#1| into a signed number, by subtracting
%   \(2^{24}\) if it is not less than \(2^{23}\).
%    \begin{macrocode}
\cs_new_protected:Npn \@@_forth_signed:N #1
  {
    \int_compare:nNnF #1 < { \c_@@_forth_mod_int / \c_two }
      { \int_sub:Nn #1 \c_@@_forth_mod_int }
  }
%    \end{macrocode}
% \end{macro}
%
% \begin{macro}[int]{\@@_forth_push_signed:n}
% \begin{macro}[aux]{\@@_forth_push_signed_aux:n}
%   Given a signed integer in the range \([-2^{24}, 2^{24}-1]\) (note
%   the extra large range), push the corresponding unsigned
%   representation onto the stack.  This simply requires adding
%   \(2^{24}\) to negative numbers.
%    \begin{macrocode}
\cs_new_protected:Npn \@@_forth_push_signed:n #1
  {
    \exp_args:Nf \@@_forth_push_signed_aux:n
      { \int_eval:n {#1} }
  }
\cs_new_protected:Npn \@@_forth_push_signed_aux:n #1
  {
    \@@_forth_push:n
      {
        \int_compare:nNnTF {#1} < \c_zero
          { #1 + \c_@@_forth_mod_int } {#1}
      }
  }
%    \end{macrocode}
% \end{macro}
% \end{macro}
%
% \begin{macro}[int]{\@@_forth_push_mod:n}
%   Given a non-negative integer, push onto the stack its residue modulo
%   \(2^{24}\).
%    \begin{macrocode}
\cs_new_protected:Npn \@@_forth_push_mod:n #1
  {
    \@@_forth_push:n
      { \int_mod:nn {#1} { \c_@@_forth_mod_int } }
  }
%    \end{macrocode}
% \end{macro}
%
% \subsubsection{Manipulating the return stack}
%
% \begin{macro}[int]{\@@_forth_return_pop_int:N}
%   Get data from the return stack.  The return stack is empty if
%   \texttt{return_here} is \texttt{return_min}.
%    \begin{macrocode}
\cs_new_protected:Npn \@@_forth_return_pop_int:N #1
  {
    \int_compare:nNnTF
      \l_@@_forth_return_here_int > \g_@@_forth_return_min_int
      {
        \int_decr:N \l_@@_forth_return_here_int
        \int_set:Nn #1
          {
            \@@_array_item:NN \g_@@_forth_array
              \l_@@_forth_return_here_int
          }
      }
      { \msg_error:nn { runner/forth } { empty-return-stack } }
  }
%    \end{macrocode}
% \end{macro}
%
% \begin{macro}[int]{\@@_forth_return_push:n}
%   Push an unsigned integer onto the return stack, at
%   \texttt{return_here}, failing if the return stack is full.
%    \begin{macrocode}
\cs_new_protected:Npn \@@_forth_return_push:n #1
  {
    \int_compare:nNnTF
      \l_@@_forth_return_here_int < \g_@@_forth_return_max_int
      {
        \@@_array_gset:Nnn \g_@@_forth_array
          \l_@@_forth_return_here_int {#1}
        \int_incr:N \l_@@_forth_return_here_int
      }
      { \msg_error:nn { runner/forth } { out-of-memory } }
  }
%    \end{macrocode}
% \end{macro}
%
% \subsubsection{Name space and code space variables}
%
% The \TeX{} code to execute Forth words is stored within \tn{toks}
% registers, labelled by so-called ``execution tokens''.  An execution
% token is a cell restricted to the range \([0, 2^{16}-1]\) or \([0,
% 2^{15}-1]\) (depending on the engine's capabilities), and its
% execution semantics are simply to run the \TeX{} code in that
% \tn{toks} register.
%
% The mapping from words to their definitions is done by giving an
% execution token pointing to the \tn{toks} with the appropriate \TeX{}
% code.  In fact, one needs a little bit more: to each word is
% associated a digit in \(\{1, 2, 3\}\) and an execution token.
% \begin{itemize}
% \item \(1\) indicates normal words, whose interpretation semantics are
%   given by the execution token, and whose compilation semantics is to
%   append the execution token to the current definition;
% \item \(2\) indicates immediate words, whose interpretation and
%   compilation semantics are identical, and given by the execution
%   token;
% \item \(3\) indicates special words such as |ABORT"|, which may have
%   arbitrary interpretation and compilation semantics: the execution
%   token gives the intepretation semantics, and the compilation
%   semantics follows in the next \tn{toks} register.
% \end{itemize}
%
% \begin{variable}{\l_@@_forth_words_prop, \g_@@_forth_core_words_prop}
%   Keys of the \texttt{words} property list are words known to Forth at
%   a given time during the execution of the code.  Keys of the
%   \texttt{core_words} global property list are words from the core
%   set, with which the \texttt{words} property list is initialized.
%
%   Values are a digit \(d \in \{1, 2, 3\}\) followed by five (decimal)
%   digits forming an execution token \(\mathrm{xt}\), which lies in the
%   range \([0, 2^{15}-1]\) for \pdfTeX{} and \XeTeX{}, and in the range
%   \([0, 2^{16}-1]\) for \LuaTeX{}.  The code to be run upon execution
%   is always found in the \tn{toks} register given by this five-digit
%   integer.  The compilation semantics is to append \(\mathrm{xt}\) to
%   the current definition if \(d = 1\), to perform \(\mathrm{xt}\) if
%   \(d = 2\), and to perform \(\mathrm{xt} + 1\) if \(d = 3\).
%    \begin{macrocode}
\prop_new:N \l_@@_forth_words_prop
\prop_new:N \g_@@_forth_core_words_prop
%    \end{macrocode}
% \end{variable}
%
% \begin{variable}{\l_@@_forth_toks_int, \g_@@_forth_core_toks_int}
%   The next free \tn{toks} register, for use to store the code
%   corresponding to execution tokens.
%    \begin{macrocode}
\int_new:N \l_@@_forth_toks_int
\int_new:N \g_@@_forth_core_toks_int
%    \end{macrocode}
% \end{variable}
%
% \begin{variable}{\g_@@_forth_init_toks_tl}
%   Token list containing part of the initialization code, currently
%   many assignments to \tn{toks} registers.
%    \begin{macrocode}
\tl_new:N \g_@@_forth_init_toks_tl
%    \end{macrocode}
% \end{variable}
%
% \subsubsection{Adding to the name space}
%
% \begin{macro}[int]
%   {
%     \@@_forth_new_core:nn,
%     \@@_forth_new_immediate_core:nn,
%     \@@_forth_new_special_core:nnn,
%     \@@_forth_new_compilation_core:nn
%   }
%    \begin{macrocode}
\cs_new_protected:Npn \@@_forth_new_core:nn #1#2
  {
    \tl_gput_right:Nx \g_@@_forth_init_toks_tl
      {
        \tex_toks:D \int_use:N \g_@@_forth_core_toks_int
          { \exp_not:n {#2} }
      }
    \prop_gput:Nnx \g_@@_forth_core_words_prop {#1}
      { \int_eval:n { 100000 + \g_@@_forth_core_toks_int } }
    \int_gincr:N \g_@@_forth_core_toks_int
  }
\cs_new_protected:Npn \@@_forth_new_immediate_core:nn #1#2
  {
    \tl_gput_right:Nx \g_@@_forth_init_toks_tl
      {
        \tex_toks:D \int_use:N \g_@@_forth_core_toks_int
          { \exp_not:n {#2} }
      }
    \prop_gput:Nnx \g_@@_forth_core_words_prop {#1}
      { \int_eval:n { 200000 + \g_@@_forth_core_toks_int } }
    \int_gincr:N \g_@@_forth_core_toks_int
  }
\cs_new_protected:Npn \@@_forth_new_special_core:nnn #1#2#3
  {
    \tl_gput_right:Nx \g_@@_forth_init_toks_tl
      {
        \tex_toks:D \int_use:N \g_@@_forth_core_toks_int
          { \exp_not:n {#2} }
        \tex_toks:D \int_eval:n { \g_@@_forth_core_toks_int + 1 }
          { \exp_not:n {#3} }
      }
    \prop_gput:Nnx \g_@@_forth_core_words_prop {#1}
      { \int_eval:n { 300000 + \g_@@_forth_core_toks_int } }
    \int_gadd:Nn \g_@@_forth_core_toks_int \c_two
  }
\cs_new_protected:Npn \@@_forth_new_compilation_core:nn #1#2
  {
    \@@_forth_new_special_core:nnn {#1}
      { \msg_error:nnn { runner/forth } { no-interpretation } {#1} }
      {#2}
  }
%    \end{macrocode}
% \end{macro}
%
% \begin{macro}[int]{\@@_forth_core_alias:nn}
%    \begin{macrocode}
\cs_new_protected:Npn \@@_forth_core_alias:nn #1#2
  {
    \prop_get:NnN \g_@@_forth_core_words_prop {#2} \l_@@_tmpa_tl
    \prop_gput:NnV \g_@@_forth_core_words_prop {#1} \l_@@_tmpa_tl
  }
%    \end{macrocode}
% \end{macro}
%
% \begin{macro}[int]{\@@_forth_new_word:nn, \@@_forth_new_word:nx}
%    \begin{macrocode}
\cs_new_protected:Npn \@@_forth_new_word:nn #1#2
  {
    \tex_toks:D \l_@@_forth_toks_int {#2}
    \prop_put:Nnx \l_@@_forth_words_prop {#1}
      { \int_eval:n { 100000 + \l_@@_forth_toks_int } }
    \int_incr:N \l_@@_forth_toks_int
  }
\cs_generate_variant:Nn \@@_forth_new_word:nn { nx }
%    \end{macrocode}
% \end{macro}
%
% \subsection{Helpers}
%
% \subsubsection{Input}
%
% \begin{macro}[int]{\@@_forth_input_spaces:}
% \begin{macro}[aux]{\@@_forth_input_spaces_aux:w}
%   Strip leading spaces from \cs{l_@@_forth_input_str} using \cs{use:nn},
%   then grab everything and store it back into the input string.
%    \begin{macrocode}
\cs_new_protected_nopar:Npn \@@_forth_input_spaces:
  {
    \exp_after:wN \use:nn
    \exp_after:wN \@@_forth_input_spaces_aux:w
    \l_@@_forth_input_str \q_stop
  }
\cs_new_protected_nopar:Npn \@@_forth_input_spaces_aux:w #1 \q_stop
  { \tl_set:Nn \l_@@_forth_input_str {#1} }
%    \end{macrocode}
% \end{macro}
% \end{macro}
%
% \begin{macro}[int]{\@@_forth_input_until:nn, \@@_forth_input_until:nN}
%   Assuming \cs{l_@@_forth_input_str} contains the trailing delimiter |#1|,
%   we split it at the first |#1| using an auxiliary function defined on
%   the fly.  This auxiliary stores the remainder back as the input
%   string, and runs the second argument of
%   \cs{@@_forth_input_until:nn}, which gets the text found as |##1|.
%   If the input contained no occurrence of |#1|, we simply need to
%   append it before-hand to get back to the case where the delimiter is
%   present.
%    \begin{macrocode}
\cs_new_protected:Npn \@@_forth_input_until:nn #1#2
  {
    \tl_if_in:NnF \l_@@_forth_input_str { #1 }
      { \tl_put_right:Nn \l_@@_forth_input_str { #1 } }
    \cs_set:Npn \@@_tmp:w ##1 #1 ##2 \q_stop
      {
        \str_set:Nn \l_@@_forth_input_str {##2}
        #2
      }
    \exp_after:wN \@@_tmp:w \l_@@_forth_input_str \q_stop
  }
\cs_new_protected:Npn \@@_forth_input_until:nN #1#2
  { \@@_forth_input_until:nn {#1} { #2 {##1} } }
%    \end{macrocode}
% \end{macro}
%
% \begin{macro}[int]{\@@_forth_input_discard:N}
%   This loop removes all occurrences of |#1| from the start of the
%   input string.  If the first token in the input string has the same
%   character code as |#1|, discard it and look for more.  Otherwise we
%   are done.
%    \begin{macrocode}
\cs_new_protected:Npn \@@_forth_input_discard:N #1
  {
    \exp_args:No \tl_if_head_eq_charcode:nNT
      \l_@@_forth_input_str #1
      {
        \str_set:Nx \l_@@_forth_input_str
          { \str_tail:N \l_@@_forth_input_str }
        \@@_forth_input_discard:N #1
      }
  }
%    \end{macrocode}
% \end{macro}
%
% \subsubsection{Numbers}
%
% \begin{macro}[int, TF]{\@@_forth_get_number:nN}
%    \begin{macrocode}
\prg_new_protected_conditional:Npnn \@@_forth_get_number:nN #1#2 { TF }
  {
    \int_set:Nn \l_@@_tmpb_int { \@@_forth_base: }
    \int_set:Nn \l_@@_tmpa_int
      { \int_min:nn { `9 } { `0 + \l_@@_tmpb_int - 1 } }
    \int_set:Nn \l_@@_tmpb_int
      { `A + \l_@@_tmpb_int - 11 }
    \bool_set_true:N \l_@@_tmpa_bool
    \tl_map_inline:nn {#1}
      {
        \int_compare:nF { `0 <= `##1 <= \l_@@_tmpa_int }
          {
            \int_compare:nF { `A <= `##1 <= \l_@@_tmpb_int }
              {
                \bool_set_false:N \l_@@_tmpa_bool
                \tl_map_break:
              }
          }
      }
    \bool_if:NTF \l_@@_tmpa_bool
      {
        \tl_set:Nx #2 { \int_from_base:nn {#1} { \@@_forth_base: } }
        \prg_return_true:
      }
      { \prg_return_false: }
  }
%    \end{macrocode}
% \end{macro}
%
% \begin{variable}{\l_@@_forth_push_fp}
%   When turning a signed number to an unsigned value for pushing onto
%   the stack, one cannot use \cs{l_@@_tmpa_int}, as it may be used by
%   the caller.  Similarly for outputting floating points modulo some
%   number.
%    \begin{macrocode}
\fp_new:N \l_@@_forth_push_fp
%    \end{macrocode}
% \end{variable}
%
% \begin{macro}[int]{\@@_forth_push_fp:n, \@@_forth_push_fp_mod:n}
%    \begin{macrocode}
\cs_new_protected:Npn \@@_forth_push_fp:n #1
  { \@@_forth_push:n { \fp_to_int:n {#1} } }
\cs_new_protected:Npn \@@_forth_push_fp_mod:n #1
  {
    \fp_set:Nn \l_@@_forth_push_fp { round0(#1) }
    \@@_forth_push_fp:n
      {
        \l_@@_forth_push_fp - \c_@@_forth_mod_int
             * round- ( \l_@@_forth_push_fp / \c_@@_forth_mod_int )
      }
  }
%    \end{macrocode}
% \end{macro}
%
% \subsubsection{Misc}
%
% \begin{variable}{\l_@@_forth_input_str}
%   When running Forth, no distinction is made between the program and
%   the input arguments to \cs{runner}, so those are combined into a
%   single string.
%    \begin{macrocode}
\str_new:N \l_@@_forth_input_str
%    \end{macrocode}
% \end{variable}
%
% \begin{variable}{\l_@@_forth_stop_bool}
%   This boolean is set to be true by the word |;| which ends
%   definitions: the loop that compiles words into the definition then
%   stops, and we return to interpreting the input.  This boolean is
%   also set true by the empty word received by the interpreter when
%   there is no input left.  This occurrence marks the end of the
%   program.
%    \begin{macrocode}
\bool_new:N \l_@@_forth_stop_bool
%    \end{macrocode}
% \end{variable}
%
% \begin{variable}{\l_@@_forth_base_address_int}
% \begin{macro}[int, EXP]{\@@_forth_base:}
%   The address at which the base of the number system is stored, and a
%   function to retreive the base, reducing it to the interval \([2,
%   36]\).
%    \begin{macrocode}
\int_new:N \l_@@_forth_base_address_int
\cs_new:Npn \@@_forth_base:
  {
    \int_min:nn { 36 }
      {
        \int_max:nn { 2 }
          { \@@_forth_data_get:N \l_@@_forth_base_address_int }
      }
  }
%    \end{macrocode}
% \end{macro}
% \end{variable}
%
% ^^A todo: implement control-flow as in the standard.
% \begin{variable}{\l_@@_forth_flow_seq}
%   Control flow stack.
%    \begin{macrocode}
\seq_new:N \l_@@_forth_flow_seq
%    \end{macrocode}
% \end{variable}
%
% \subsection{Core words}
%
% \subsubsection{Definition words}
%
% \begin{variable}{\l_@@_forth_def_tl}
%   The definition being built.
%    \begin{macrocode}
\tl_new:N \l_@@_forth_def_tl
%    \end{macrocode}
% \end{variable}
%
% \begin{macro}[int]{\@@_forth_def_put_right:n, \@@_forth_def_put_right:x}
% \begin{macro}[int]{\@@_forth_def_put_right_x:n}
%   Helpers to add material to the definition, with various types of
%   expansion.
%    \begin{macrocode}
\cs_new_protected:Npn \@@_forth_def_put_right:n #1
  { \tl_put_right:Nn \l_@@_forth_def_tl { \exp_not:n {#1} } }
\cs_generate_variant:Nn \@@_forth_def_put_right:n { x }
\cs_new_protected:Npn \@@_forth_def_put_right_x:n #1
  { \tl_put_right:Nn \l_@@_forth_def_tl {#1} }
%    \end{macrocode}
% \end{macro}
% \end{macro}
%
% \begin{variable}{\l_@@_forth_def_nesting_int}
%   Since conditionals add unmatched braces to the definition, we must
%   ensure that at the end of the day there is no extra open or closed
%   brace.
%    \begin{macrocode}
\int_new:N \l_@@_forth_def_nesting_int
%    \end{macrocode}
% \end{variable}
%
% \begin{variable}{\l_@@_forth_def_name_str}
%   The last word that was defined: this is set just after the word's
%   meaning is changed, after the definition is fully read.
%    \begin{macrocode}
\tl_new:N \l_@@_forth_def_name_str
%    \end{macrocode}
% \end{variable}
%
% \begin{macro}[int]{CONSTANT}
%    \begin{macrocode}
\@@_forth_new_core:nn { CONSTANT }
  {
    \@@_forth_pop_int:N \l_@@_tmpa_int
    \@@_forth_input_spaces:
    \@@_forth_input_until:nn { ~ }
      {
        \@@_forth_new_word:nx {#1}
          { \@@_forth_push:n { \int_use:N \l_@@_tmpa_int } }
      }
  }
%    \end{macrocode}
% \end{macro}
%
% \begin{macro}[int]{VARIABLE}
%    \begin{macrocode}
\@@_forth_new_core:nn { VARIABLE }
  {
    \@@_forth_input_spaces:
    \@@_forth_input_until:nn { ~ }
      {
        \int_compare:nNnTF
          \l_@@_forth_data_here_int < \l_@@_forth_data_text_int
          {
            \@@_forth_new_word:nx {#1}
              {
                \@@_forth_push:n
                  { \int_use:N \l_@@_forth_data_here_int }
              }
            \int_incr:N \l_@@_forth_data_here_int
          }
          { \msg_error:nn { runner/forth } { out-of-memory } }
      }
  }
%    \end{macrocode}
% \end{macro}
%
% ^^A todo: what is the compilation semantics of the colon word?
% \begin{macro}[int]{:}
%   Read a word, and feed it to \cs{@@_forth_def:n}, responsible for all
%   the work.
%    \begin{macrocode}
\@@_forth_new_core:nn { : }
  {
    \@@_forth_input_spaces:
    \@@_forth_input_until:nN { ~ } \@@_forth_def:n
  }
%    \end{macrocode}
% \end{macro}
%
% ^^A todo: robustify the flow_seq against bad input (nesting).
% ^^A todo: improve error recovery with missing THEN.
% \begin{macro}[int]{\@@_forth_def:n}
%   Store in the control-flow stack the definition being built, if any,
%   and empty \cs{l_@@_forth_def_tl}.  Then switch to compiling state,
%   looping until something (typically, the word |;|) makes the ``stop''
%   boolean true.  We then set the boolean back to be false: otherwise,
%   the interpreter loop would end.  A new non-immediate word is then
%   defined with the name |#1| and the definition that was collected.
%   Finally, the name |#1| is store (for |IMMEDIATE|) and the former
%   value of \cs{l_@@_forth_def_tl} is restored.
%    \begin{macrocode}
\cs_new_protected:Npn \@@_forth_def:n #1
  {
    \seq_push:NV \l_@@_forth_flow_seq \l_@@_forth_def_tl
    \tl_clear:N \l_@@_forth_def_tl
    \bool_until_do:Nn \l_@@_forth_stop_bool
      {
        \@@_forth_input_spaces:
        \@@_forth_input_until:nN { ~ } \@@_forth_compile:n
      }
    \bool_set_false:N \l_@@_forth_stop_bool
    \int_compare:nNnF \l_@@_forth_def_nesting_int = \c_zero
      {
        \msg_error:nn { runner/forth } { too-many-ifs }
        \prg_replicate:nn \l_@@_forth_def_nesting_int
          { \@@_forth_def_put_right_x:n { \if_false: { \fi: } } }
      }
    \@@_forth_new_word:nx {#1} { \l_@@_forth_def_tl }
    \str_set:Nn \l_@@_forth_def_name_str {#1}
    \seq_pop:NN \l_@@_forth_flow_seq \l_@@_forth_def_tl
  }
%    \end{macrocode}
% \end{macro}
%
% \begin{macro}[int]{\@@_forth_compile:n}
%   This is very similar to \cs{@@_forth_interpret:n}.  Try to find the
%   word in the dictionary.  If it is present, split its value into \(1
%   + 5\) digits: if the first is \(1\), simply add to the current
%   definition; if the first is \(2\) perform the execution semantics,
%   and if it is \(3\), the word is a special word, and we perform the
%   compilation semantics.  Otherwise, we try to interpret the word as a
%   number.
%    \begin{macrocode}
\cs_new_protected:Npn \@@_forth_compile:n #1
  {
    \prop_get:NnNTF \l_@@_forth_words_prop {#1} \l_@@_tmpa_tl
      { \exp_after:wN \@@_forth_compile:Nw \l_@@_tmpa_tl \q_stop }
      {
        \@@_forth_get_number:nNTF {#1} \l_@@_tmpa_tl
          {
            \@@_forth_def_put_right:x
              { \@@_forth_push:n { \l_@@_tmpa_tl } }
          }
          { \msg_error:nnn { runner/forth } { unknown-word } {#1} }
      }
  }
\cs_new_protected:Npn \@@_forth_compile:Nw #1#2 \q_stop
  {
    \int_case:nnn {#1}
      {
        { 1 }
          {
            \@@_forth_def_put_right:n
              { \tex_the:D \tex_toks:D #2 \scan_stop: }
          }
        { 2 } { \tex_the:D \tex_toks:D #2 \scan_stop: }
      }
      { \tex_the:D \tex_toks:D \int_eval:n { #2 + 1 } \scan_stop: }
  }
%    \end{macrocode}
% \end{macro}
%
% \begin{macro}[int]{;}
%   All the finishing touches for a definition are done by the
%   implementation of the colon word, and here we only need to stop the
%   compile loop by setting a boolean.
%    \begin{macrocode}
\@@_forth_new_compilation_core:nn { ; }
  { \bool_set_true:N \l_@@_forth_stop_bool }
%    \end{macrocode}
% \end{macro}
%
% \begin{macro}[int]{IMMEDIATE}
%    \begin{macrocode}
\@@_forth_new_core:nn { IMMEDIATE }
  {
    \prop_get:NVNTF
      \l_@@_forth_words_prop
      \l_@@_forth_def_name_str
      \l_@@_tmpa_tl
      {
        \tl_set:Nx \l_@@_tmpa_tl
          { \int_eval:n { 100000 + \l_@@_tmpa_tl } }
        \prop_put:NVV
          \l_@@_forth_words_prop
          \l_@@_forth_def_name_str
          \l_@@_tmpa_tl
      }
      {
        \msg_error:nnx { runner/forth } { no-def-immediate }
          { \l_@@_forth_def_name_str }
      }
  }
%    \end{macrocode}
% \end{macro}
%
% \begin{macro}[int]{LITERAL}
%    \begin{macrocode}
\@@_forth_new_compilation_core:nn { LITERAL }
  {
    \@@_forth_pop_int:N \l_@@_tmpa_int
    \@@_forth_def_put_right:x
      { \@@_forth_push:n { \int_use:N \l_@@_tmpa_int } }
  }
%    \end{macrocode}
% \end{macro}
%
% \begin{macro}[int]{POSTPONE}
% \begin{macro}[aux, EXP]{\@@_forth_postpone:Nw}
%    \begin{macrocode}
\@@_forth_new_compilation_core:nn { POSTPONE }
  {
    \@@_forth_input_spaces:
    \@@_forth_input_until:nn { ~ }
      {
        \prop_get:NnN \l_@@_forth_words_prop {#1} \l_@@_tmpa_tl
        \@@_forth_def_put_right:x
          {
            \exp_after:wN \@@_forth_postpone:Nw
              \l_@@_tmpa_tl \q_stop
          }
      }
  }
\cs_new:Npn \@@_forth_postpone:Nw #1#2 \q_stop
  {
    \int_case:nnn {#1}
      {
        { 1 }
          {
            \@@_forth_def_put_right:n
              { \exp_not:N \tex_the:D \tex_toks:D #2 \scan_stop: }
          }
        { 2 } { \exp_not:N \tex_the:D \tex_toks:D #2 \scan_stop: }
      }
      {
        \exp_not:N \tex_the:D \tex_toks:D
          \int_eval:n { #2 + 1 } \scan_stop:
      }
  }
%    \end{macrocode}
% \end{macro}
% \end{macro}
%
% \begin{macro}[int]{RECURSE}
%   When the definition is expanded prior to the assignment,
%   \cs{l_@@_forth_toks_int} is the value of the \tn{toks} register
%   which will contain the code.  Recursion involves calling this code
%   from within itself.
%    \begin{macrocode}
\@@_forth_new_compilation_core:nn { RECURSE }
  {
    \@@_forth_def_put_right_x:n
      {
        \exp_not:N \tex_the:D \tex_toks:D
        \int_use:N \l_@@_forth_toks_int \scan_stop:
      }
  }
%    \end{macrocode}
% \end{macro}
%
% \subsubsection{Execution words}
%
% \begin{macro}[int]{'}
%   Push onto the data stack the execution token associated to a word
%   (read from the input).  The word is read from the input after
%   removing all leading spaces.
%    \begin{macrocode}
\@@_forth_new_core:nn { ' }
  {
    \@@_forth_input_spaces:
    \@@_forth_input_until:nn { ~ }
      {
        \prop_get:NnN \l_@@_forth_words_prop {#1} \l_@@_tmpa_tl
        \@@_forth_push:n
          { \exp_after:wN \use_none:n \l_@@_tmpa_tl }
      }
  }
%    \end{macrocode}
% \end{macro}
%
% \begin{macro}[int]{[']}
%   Read a word, find it in the dictionary (prop), and find its
%   execution token (the \cs{use_none:n} construction).  Then add to the
%   current definition some code to place this execution token on the
%   stack.
%    \begin{macrocode}
\@@_forth_new_compilation_core:nn { ['] }
  {
    \@@_forth_input_spaces:
    \@@_forth_input_until:nn { ~ }
      {
        \prop_get:NnN \l_@@_forth_words_prop {#1} \l_@@_tmpa_tl
        \@@_forth_def_put_right:x
          {
            \@@_forth_push:n
              { \exp_after:wN \use_none:n \l_@@_tmpa_tl }
          }
      }
  }
%    \end{macrocode}
% \end{macro}
%
% ^^A todo: take (tmpa mod 2^24).
% \begin{macro}[int]{EXECUTE}
%   Pop from the data stack an execution token.  Perform it.
%    \begin{macrocode}
\@@_forth_new_core:nn { EXECUTE }
  {
    \@@_forth_pop_int:N \l_@@_tmpa_int
    \tex_the:D \tex_toks:D \l_@@_tmpa_int
  }
%    \end{macrocode}
% \end{macro}
%
% \subsubsection{Stack words}
%
% \begin{macro}[int]{DEPTH}
%    \begin{macrocode}
\@@_forth_new_core:nn { DEPTH }
  {
    \@@_forth_push:n
      { \l_@@_forth_stack_here_int - \g_@@_forth_stack_min_int }
  }
%    \end{macrocode}
% \end{macro}
%
% \begin{macro}[int]{>R}
%   Pop from the data stack onto the return stack.
%    \begin{macrocode}
\@@_forth_new_compilation_core:nn { >R }
  {
    \@@_forth_pop_int:N \l_@@_tmpa_int
    \@@_forth_return_push:n \l_@@_tmpa_int
  }
%    \end{macrocode}
% \end{macro}
%
% ^^A todo: |R>| and |R@| can blow up if return stack contains non-numbers.
% \begin{macro}[int]{R>}
%   Pop from the return stack onto the data stack.
%    \begin{macrocode}
\@@_forth_new_core:nn { R> }
  {
    \@@_forth_return_pop_int:N \l_@@_tmpa_int
    \@@_forth_push:n { \l_@@_tmpa_int }
  }
%    \end{macrocode}
% \end{macro}
%
% ^^A todo: optimize.
% \begin{macro}[int]{R@}
%   Copy from the return stack onto the data stack.
%    \begin{macrocode}
\@@_forth_new_core:nn { R@ }
  {
    \@@_forth_return_pop_int:N \l_@@_tmpa_int
    \@@_forth_return_push:n { \l_@@_tmpa_int }
    \@@_forth_push:n { \l_@@_tmpa_int }
  }
%    \end{macrocode}
% \end{macro}
%
% \begin{macro}[int]{?DUP}
%   Duplicate top of stack if non-zero.
%    \begin{macrocode}
\@@_forth_new_core:nn { ?DUP }
  {
    \@@_forth_pop_int:N \l_@@_tmpa_int
    \int_compare:nNnTF \l_@@_tmpa_int = \c_zero
      { \@@_forth_push:n { \c_zero } }
      {
        \@@_forth_push:nn
          { \l_@@_tmpa_int } { \l_@@_tmpa_int }
      }
  }
%    \end{macrocode}
% \end{macro}
%
% \begin{macro}[int]{DROP, 2DROP}
%   Pop stack, once or twice.
%    \begin{macrocode}
\@@_forth_new_core:nn { DROP }
  { \@@_forth_pop_int:N \l_@@_tmpa_int }
\@@_forth_new_core:nn { 2DROP }
  { \@@_forth_pop_int:NN \l_@@_tmpa_int \l_@@_tmpb_int }
%    \end{macrocode}
% \end{macro}
%
% \begin{macro}[int]{DUP, 2DUP}
%   Turn \(x\) to \(x, x\), or \(x_1, x_2\) to \(x_1, x_2, x_1, x_2\).
%    \begin{macrocode}
\@@_forth_new_core:nn { DUP }
  {
    \@@_forth_pop_int:N \l_@@_tmpa_int
    \@@_forth_push:nn { \l_@@_tmpa_int } { \l_@@_tmpa_int }
  }
\@@_forth_new_core:nn { 2DUP }
  {
    \@@_forth_pop_int:NN \l_@@_tmpa_int \l_@@_tmpb_int
    \@@_forth_push:nnnn
      { \l_@@_tmpa_int } { \l_@@_tmpb_int }
      { \l_@@_tmpa_int } { \l_@@_tmpb_int }
  }
%    \end{macrocode}
% \end{macro}
%
% \begin{macro}[int]{OVER, 2OVER}
%   Turn \(x_1, x_2\) to \(x_1, x_2, x_1\), or \(x_1, x_2, x_3, x_4\) to
%   \(x_1, x_2, x_3, x_4, x_1, x_2\).
%    \begin{macrocode}
\@@_forth_new_core:nn { OVER }
  {
    \@@_forth_pop_int:NN \l_@@_tmpa_int \l_@@_tmpb_int
    \@@_forth_push:nnn
      { \l_@@_tmpa_int } { \l_@@_tmpb_int } { \l_@@_tmpa_int }
  }
\@@_forth_new_core:nn { 2OVER }
  {
    \@@_forth_pop_int:NNNN
      \l_@@_tmpa_int \l_@@_tmpb_int
      \l_@@_tmpc_int \l_@@_tmpd_int
    \@@_forth_push:nnnnnn
      { \l_@@_tmpa_int } { \l_@@_tmpb_int }
      { \l_@@_tmpc_int } { \l_@@_tmpd_int }
      { \l_@@_tmpa_int } { \l_@@_tmpb_int }
  }
%    \end{macrocode}
% \end{macro}
%
% \begin{macro}[int]{ROT}
%   Turn \(x_1, x_2, x_3\) to \(x_2, x_3, x_1\).
%    \begin{macrocode}
\@@_forth_new_core:nn { ROT }
  {
    \@@_forth_pop_int:NNN
      \l_@@_tmpa_int \l_@@_tmpb_int \l_@@_tmpc_int
    \@@_forth_push:nnn
      { \l_@@_tmpb_int } { \l_@@_tmpc_int } { \l_@@_tmpa_int }
  }
%    \end{macrocode}
% \end{macro}
%
% \begin{macro}[int]{SWAP, 2SWAP}
%   Turn \(x_1, x_2\) to \(x_2, x_1\), or \(x_1, x_2, x_3, x_4\) to
%   \(x_3, x_4, x_1, x_2\).
%    \begin{macrocode}
\@@_forth_new_core:nn { SWAP }
  {
    \@@_forth_pop_int:NN \l_@@_tmpa_int \l_@@_tmpb_int
    \@@_forth_push:nn { \l_@@_tmpb_int } { \l_@@_tmpa_int }
  }
\@@_forth_new_core:nn { 2SWAP }
  {
    \@@_forth_pop_int:NNNN
      \l_@@_tmpa_int \l_@@_tmpb_int
      \l_@@_tmpc_int \l_@@_tmpd_int
    \@@_forth_push:nn
      { \l_@@_tmpc_int } { \l_@@_tmpd_int }
      { \l_@@_tmpa_int } { \l_@@_tmpb_int }
  }
%    \end{macrocode}
% \end{macro}
%
% \subsubsection{Comparison words}
%
% \begin{macro}[int]{0<}
%   If \(a<0\) push true, otherwise false.
%    \begin{macrocode}
\@@_forth_new_core:nn { 0< }
  {
    \@@_forth_pop_int:N \l_@@_tmpa_int
    \int_compare:nNnTF \l_@@_tmpa_int < \c_zero
      { \@@_forth_push_true: }
      { \@@_forth_push_false: }
  }
%    \end{macrocode}
% \end{macro}
%
% \begin{macro}[int]{0=}
%   If \(a\) is \(0\), push true, otherwise false.
%    \begin{macrocode}
\@@_forth_new_core:nn { 0= }
  {
    \@@_forth_pop_int:N \l_@@_tmpa_int
    \int_compare:nNnTF \l_@@_tmpa_int = \c_zero
      { \@@_forth_push_true: }
      { \@@_forth_push_false: }
  }
%    \end{macrocode}
% \end{macro}
%
% \begin{macro}[int]{<, =, >}
%   Comparisons.  Push true or false.
%    \begin{macrocode}
\@@_forth_new_core:nn { < }
  {
    \@@_forth_pop_int:NN \l_@@_tmpa_int \l_@@_tmpb_int
    \int_compare:nNnTF \l_@@_tmpa_int < \l_@@_tmpb_int
      { \@@_forth_push_true: }
      { \@@_forth_push_false: }
  }
\@@_forth_new_core:nn { = }
  {
    \@@_forth_pop_int:NN \l_@@_tmpa_int \l_@@_tmpb_int
    \int_compare:nNnTF \l_@@_tmpa_int = \l_@@_tmpb_int
      { \@@_forth_push_true: }
      { \@@_forth_push_false: }
  }
\@@_forth_new_core:nn { > }
  {
    \@@_forth_pop_int:NN \l_@@_tmpa_int \l_@@_tmpb_int
    \int_compare:nNnTF \l_@@_tmpa_int > \l_@@_tmpb_int
      { \@@_forth_push_true: }
      { \@@_forth_push_false: }
  }
%    \end{macrocode}
% \end{macro}
%
% \subsubsection{Arithmetic words}
%
% \begin{macro}[int]{1+, 1-}
%   Increment or decrement the top of the stack.
%    \begin{macrocode}
\@@_forth_new_core:nn { 1+ }
  {
    \@@_forth_pop_int:N \l_@@_tmpa_int
    \@@_forth_push:n { \l_@@_tmpa_int + 1 }
  }
\@@_forth_new_core:nn { 1- }
  {
    \@@_forth_pop_int:N \l_@@_tmpa_int
    \@@_forth_push:n { \l_@@_tmpa_int - 1 }
  }
%    \end{macrocode}
% \end{macro}
%
% \begin{macro}[int]{2*, 2/}
%   Multiply or divide the top of the stack by~\(2\).  Neither \TeX{}'s
%   rounding division nor \LaTeX3's truncating division do the right
%   thing there, as we want floored division (\((-1)/2=-1\)), so we
%   distinguish the even and odd cases.
%    \begin{macrocode}
\@@_forth_new_core:nn { 2* }
  {
    \@@_forth_pop_int:N \l_@@_tmpa_int
    \@@_forth_push:n { \l_@@_tmpa_int * 2 }
  }
\@@_forth_new_core:nn { 2/ }
  {
    \@@_forth_pop_int:N \l_@@_tmpa_int
    \int_if_even:nTF { \l_@@_tmpa_int }
      { \@@_forth_push:n { \l_@@_tmpa_int / 2 } }
      { \@@_forth_push:n { (\l_@@_tmpa_int - 1) / 2 } }
  }
%    \end{macrocode}
% \end{macro}
%
% \begin{macro}[int]{ABS}
%   Get a signed integer from the top of the stack, and change it to its
%   absolute value.
%    \begin{macrocode}
\@@_forth_new_core:nn { ABS }
  {
    \@@_forth_pop_int:N \l_@@_tmpa_int
    \@@_forth_signed:N \l_@@_tmpa_int
    \@@_forth_push:n { \int_abs:n { \l_@@_tmpa_int } }
  }
%    \end{macrocode}
% \end{macro}
%
% \begin{macro}[int]{MAX, MIN}
%   Get two signed integers from the stack, and push the
%   biggest/smallest back onto the stack.
%    \begin{macrocode}
\@@_forth_new_core:nn { MAX }
  {
    \@@_forth_pop_int:NN \l_@@_tmpa_int \l_@@_tmpb_int
    \@@_forth_signed:N \l_@@_tmpa_int
    \@@_forth_signed:N \l_@@_tmpb_int
    \@@_forth_push_signed:n
      { \int_max:nn { \l_@@_tmpa_int } { \l_@@_tmpb_int } }
  }
\@@_forth_new_core:nn { MIN }
  {
    \@@_forth_pop_int:NN \l_@@_tmpa_int \l_@@_tmpb_int
    \@@_forth_signed:N \l_@@_tmpa_int
    \@@_forth_signed:N \l_@@_tmpb_int
    \@@_forth_push_signed:n
      { \int_min:nn { \l_@@_tmpa_int } { \l_@@_tmpb_int } }
  }
%    \end{macrocode}
% \end{macro}
%
% \begin{macro}[int]{NEGATE}
%   Get a signed integer from the top of the stack, and change it to its
%   opposite.
%    \begin{macrocode}
\@@_forth_new_core:nn { ABS }
  {
    \@@_forth_pop_int:N \l_@@_tmpa_int
    \@@_forth_signed:N \l_@@_tmpa_int
    \@@_forth_push_signed:n { - \l_@@_tmpa_int }
  }
%    \end{macrocode}
% \end{macro}
%
% \begin{macro}[int]{-, +}
%   Get two integers from the data stack, sum or take the difference,
%   and push the result onto the stack.  The integers can be signed or
%   unsigned, but are given to us as unsigned values in \([0,
%   2^{24}-1]\), hence there is an ambiguity by \(2^{24}\).  The
%   difference of two unsigned values lies in \([-2^{24}+1, 2^{24}-1]\),
%   and can be brought back to \([0, 2^{24}-1]\) by adding \(2^{24}\) to
%   negative numbers, as \cs{@@_forth_push_signed:n} does.  The sum,
%   shifted by \(2^{24}\), lies in \([-2^{24}, 2^{24}-2]\), and is also
%   appropriate input for the \texttt{push_signed} function.
%    \begin{macrocode}
\@@_forth_new_core:nn { - }
  {
    \@@_forth_pop_int:NN \l_@@_tmpa_int \l_@@_tmpb_int
    \@@_forth_push_signed:n
      { \l_@@_tmpa_int - \l_@@_tmpb_int }
  }
\@@_forth_new_core:nn { + }
  {
    \@@_forth_pop_int:NN \l_@@_tmpa_int \l_@@_tmpb_int
    \@@_forth_push_signed:n
      {
        \l_@@_tmpa_int + \l_@@_tmpb_int
        - \c_@@_forth_mod_int
      }
  }
%    \end{macrocode}
% \end{macro}
%
% \begin{macro}[int]{*}
%   Multiply two integers, then push the result (modulo \(2^{24}\)) onto
%   the stack.  To avoid \TeX{} overflow, we manipulate the numbers as
%   floating points, computing \(a\times b - 2^{24}\times\bigl\lfloor
%   a\times b / 2^{24}\bigr\rfloor\).
%    \begin{macrocode}
\@@_forth_new_core:nn { * }
  {
    \@@_forth_pop_int:NN \l_@@_tmpa_int \l_@@_tmpb_int
    \@@_forth_push_fp_mod:n
      { \l_@@_tmpa_int * \l_@@_tmpb_int }
  }
%    \end{macrocode}
% \end{macro}
%
% ^^A todo: division by zero?
% \begin{macro}[int]{/}
%   Pop two integers from the stack.  The input is assumed signed, so we
%   subtract \(2^{24}\) if the integers are \(2^{23}\) or more.  Then
%   perform the division, and push the result (converted back to being
%   unsigned) onto the stack.
%    \begin{macrocode}
\@@_forth_new_core:nn { / }
  {
    \@@_forth_pop_int:NN \l_@@_tmpa_int \l_@@_tmpb_int
    \@@_forth_signed:N \l_@@_tmpa_int
    \@@_forth_signed:N \l_@@_tmpb_int
    \@@_forth_push_signed:n
      {
        \int_div_truncate:nn
          { \l_@@_tmpa_int } { \l_@@_tmpb_int }
      }
  }
%    \end{macrocode}
% \end{macro}
%
% ^^A todo: division by zero?
% \begin{macro}[int]{MOD}
%   Pop two signed integers from the stack.  Push the remainder of \(a\)
%   divided by \(b\) onto the stack.
%    \begin{macrocode}
\@@_forth_new_core:nn { MOD }
  {
    \@@_forth_pop_int:NN \l_@@_tmpa_int \l_@@_tmpb_int
    \@@_forth_signed:N \l_@@_tmpa_int
    \@@_forth_signed:N \l_@@_tmpb_int
    \@@_forth_push_signed:n
      { \int_mod:nn { \l_@@_tmpa_int } { \l_@@_tmpb_int } }
  }
%    \end{macrocode}
% \end{macro}
%
% ^^A todo: division by zero?
% \begin{macro}[int]{/MOD}
%   Get two signed integers from the stack, then perform the division,
%   and put the remainder, then the quotient, on the stack.  Both
%   quotient and remainder remain in the range \([-2^{23}, 2^{23}]\)
%   (the upper bound happens when computing \((-2^{23})/(-1)\)), and are
%   brought back to an unsigned form before pushing onto the stack.
%    \begin{macrocode}
\@@_forth_new_core:nn { /MOD }
  {
    \@@_forth_pop_int:NN \l_@@_tmpa_int \l_@@_tmpb_int
    \@@_forth_signed:N \l_@@_tmpa_int
    \@@_forth_signed:N \l_@@_tmpb_int
    \@@_forth_push_signed:n
      { \int_mod:nn { \l_@@_tmpa_int } { \l_@@_tmpb_int } }
    \@@_forth_push_signed:n
      {
        \int_div_truncate:nn
          { \l_@@_tmpa_int } { \l_@@_tmpb_int }
      }
  }
%    \end{macrocode}
% \end{macro}
%
% ^^A todo: division by zero?
% ^^A todo: bug because of double rounding.  Later, fp will provide mod.
% ^^A todo: test whether I just completely screwed up.
% \begin{macro}[int]{*/, */MOD}
%   Those two words have a lot of code in common, and start by popping
%   three signed integers from the stack.  Then compute \(a\times b/c\)
%   using floating points, rounding it towards zero.  For |*/| this is
%   the end, and the result is output (after suitably reducing its
%   range).  For |*/MOD|, use that quotient to compute the remainder.
%    \begin{macrocode}
\@@_forth_new_core:nn { */ }
  {
    \@@_forth_pop_int:NNN
      \l_@@_tmpa_int \l_@@_tmpb_int \l_@@_tmpc_int
    \@@_forth_signed:N \l_@@_tmpa_int
    \@@_forth_signed:N \l_@@_tmpb_int
    \@@_forth_signed:N \l_@@_tmpc_int
    \@@_forth_push_fp_mod:n
      { \l_@@_tmpa_int * \l_@@_tmpb_int / \l_@@_tmpc_int }
  }
\@@_forth_new_core:nn { */MOD }
  {
    \@@_forth_pop_int:NNN
      \l_@@_tmpa_int \l_@@_tmpb_int \l_@@_tmpc_int
    \@@_forth_signed:N \l_@@_tmpa_int
    \@@_forth_signed:N \l_@@_tmpb_int
    \@@_forth_signed:N \l_@@_tmpc_int
    \fp_set:Nn \l_@@_tmpa_fp
      {
        round0 ( \l_@@_tmpa_int * \l_@@_tmpb_int
        / \l_@@_tmpc_int )
      }
    \@@_forth_push_fp_mod:n
      {
        \l_@@_tmpa_int * \l_@@_tmpb_int
        - \l_@@_tmpc_int * \l_@@_tmpa_fp
      }
    \@@_forth_push_fp_mod:n { \l_@@_tmpa_fp }
  }
%    \end{macrocode}
% \end{macro}
%
% \subsubsection{Storage words}
%
% ^^A todo: Simplify how |ALIGNED| triggers an error upon empty stack
% \begin{macro}[int]{ALIGN, ALIGNED}
%   Since all addresses are aligned, |ALIGN| does nothing, |ALIGNED|
%   simply pops and pushes back the same value.
%    \begin{macrocode}
\@@_forth_new_core:nn { ALIGN } { }
\@@_forth_new_core:nn { ALIGNED }
  {
    \@@_forth_pop_int:N \l_@@_tmpa_int
    \@@_forth_push:n \l_@@_tmpa_int
  }
%    \end{macrocode}
% \end{macro}
%
% \begin{macro}[int]{ALLOT}
%   Shift the data pointer by the value at the top of the stack
%    \begin{macrocode}
\@@_forth_new_core:nn { ALLOT }
  {
    \@@_forth_pop_int:N \l_@@_tmpa_int
    \int_add:Nn \l_@@_forth_data_here_int \l_@@_tmpa_int
    \int_compare:nNnTF
      \l_@@_forth_data_here_int < \g_@@_forth_data_min_int
      {
        \msg_error:nn { runner/forth } { out-of-bounds }
        \int_set_eq:NN \l_@@_forth_data_here_int
          \g_@@_forth_data_min_int
      }
      {
        \int_compare:nNnT
          \l_@@_forth_data_here_int > \l_@@_forth_data_text_int
          {
            \msg_error:nn { runner/forth } { out-of-memory }
            \int_set_eq:NN \l_@@_forth_data_here_int
              \l_@@_forth_data_text_int
          }
      }
  }
%    \end{macrocode}
% \end{macro}
%
% \begin{macro}[int]{FILL}
%   Fill the addresses from \(a\) to \(a+b-1\) with the value \(c\).
%    \begin{macrocode}
\@@_forth_new_core:nn { FILL }
  {
    \@@_forth_pop_int:NNN
      \l_@@_tmpa_int \l_@@_tmpb_int \l_@@_tmpc_int
    \int_step_inline:nnnn
      { \l_@@_tmpa_int }
      { 1 }
      { \l_@@_tmpa_int + \l_@@_tmpb_int - 1 }
      { \@@_forth_data_gset:nn {#1} { \l_@@_tmpc_int } }
  }
%    \end{macrocode}
% \end{macro}
%
% \begin{macro}[int]{HERE}
%    \begin{macrocode}
\@@_forth_new_core:nn { HERE }
  { \@@_forth_push:n { \l_@@_forth_data_here_int } }
%    \end{macrocode}
% \end{macro}
%
% \begin{macro}[int]{!}
%   Store value at a given address.
%    \begin{macrocode}
\@@_forth_new_core:nn { ! }
  {
    \@@_forth_pop_int:NN \l_@@_tmpa_int \l_@@_tmpb_int
    \@@_forth_data_gset:nn
      { \l_@@_tmpb_int } { \l_@@_tmpa_int }
  }
%    \end{macrocode}
% \end{macro}
%
% \begin{macro}[int]{2!}
%   Store values at a given address and at the next cell.
%    \begin{macrocode}
\@@_forth_new_core:nn { 2! }
  {
    \@@_forth_pop_int:NNN
      \l_@@_tmpa_int \l_@@_tmpb_int \l_@@_tmpc_int
    \@@_forth_data_gset:nn
      { \l_@@_tmpc_int } { \l_@@_tmpb_int }
    \@@_forth_data_gset:nn
      { \l_@@_tmpc_int + 1 } { \l_@@_tmpa_int }
  }
%    \end{macrocode}
% \end{macro}
%
% \begin{macro}[int]{+!}
%   Sum \(a\) and the value at \(b\) into the integer
%   \cs{l_@@_tmpa_int}, then store that into the address at \(b\).
%    \begin{macrocode}
\@@_forth_new_core:nn { +! }
  {
    \@@_forth_pop_int:NN \l_@@_tmpa_int \l_@@_tmpb_int
    \int_add:Nn \l_@@_tmpa_int
      { \@@_forth_data_get:N \l_@@_tmpb_int }
    \int_compare:nNnF \l_@@_tmpa_int < \c_@@_forth_mod_int
      { \int_sub:Nn \l_@@_tmpa_int \c_@@_forth_mod_int }
    \@@_forth_data_gset:nn
      { \l_@@_tmpb_int } { \l_@@_tmpa_int }
  }
%    \end{macrocode}
% \end{macro}
%
% \begin{macro}[int]+,+
%   Store value in a newly allocated cell, given by the data-space
%   pointer ``\texttt{here}''.
%    \begin{macrocode}
\@@_forth_new_core:nn { , }
  {
    \@@_forth_pop_int:N \l_@@_tmpa_int
    \@@_forth_put_here:n { \l_@@_tmpa_int }
  }
%    \end{macrocode}
% \end{macro}
%
% \begin{macro}[int]{@}
%   Pop an address from the stack, and push the value \(x\) of the
%   corresponding memory cell.
%    \begin{macrocode}
\@@_forth_new_core:nn { @ }
  {
    \@@_forth_pop_int:N \l_@@_tmpa_int
    \@@_forth_push:n
      { \@@_forth_data_get:N \l_@@_tmpa_int }
  }
%    \end{macrocode}
% \end{macro}
%
% \begin{macro}[int]{2@}
%   Find an address on the stack, and fetch the value \(x_2\) of the
%   corresponding memory cell, and the next, \(x_1\).  The value \(x_2\)
%   (at the given address) ends up at the top of the stack.
%    \begin{macrocode}
\@@_forth_new_core:nn { 2@ }
  {
    \@@_forth_pop_int:N \l_@@_tmpa_int
    \@@_forth_push:nn
      { \@@_forth_data_get:n { \l_@@_tmpa_int + 1 } }
      { \@@_forth_data_get:N \l_@@_tmpa_int }
  }
%    \end{macrocode}
% \end{macro}
%
% \begin{macro}[int]{C!, C@}
% \begin{macro}[int]+C,+
%   Since cells are one character wide, storing at a character-aligned
%   address, or at an aligned adress, is the same.
%    \begin{macrocode}
\@@_forth_core_alias:nn { C! } { ! }
\@@_forth_core_alias:nn { C, } { , }
\@@_forth_core_alias:nn { C@ } { @ }
%    \end{macrocode}
% \end{macro}
% \end{macro}
%
% \begin{macro}[int]{CELL+, CELLS, CHAR+, CHARS}
%   Since the cell size, the character size and the address sizes are
%   identical, |CELL+| and |CHAR+| simply add~\(1\), like |1+|, and
%   |CELLS| and |CHARS| do nothing to the top of the stack (which must
%   be present), just like |ALIGNED|.
%    \begin{macrocode}
\@@_forth_core_alias:nn { CELL+ } { 1+ }
\@@_forth_core_alias:nn { CHAR+ } { 1+ }
\@@_forth_core_alias:nn { CELLS } { ALIGNED }
\@@_forth_core_alias:nn { CHARS } { ALIGNED }
%    \end{macrocode}
% \end{macro}
%
% \subsubsection{Words for display, input, and strings}
%
% ^^A todo: this fails for unsigned double-cell integers: too large!
% \begin{macro}[int]{.}
%   Get a value from the data stack, convert it to the base, which can
%   be altered through the Forth keyword |BASE|.  Trailing space.
%    \begin{macrocode}
\@@_forth_new_core:nn { . }
  {
    \@@_forth_pop_int:N \l_@@_tmpa_int
    \@@_output:x
      { \int_to_base:nn { \l_@@_tmpa_int } { \@@_forth_base: } ~ }
  }
%    \end{macrocode}
% \end{macro}
%
% \begin{macro}[int]+."+
%   The dot-quote word reads the input until a double quote.  It has no
%   interpretation semantics.  Its compilation semantics is to add code
%   to the current definition that display what it read.
%    \begin{macrocode}
\@@_forth_new_compilation_core:nn { ." }
  {
    \@@_forth_input_until:nn "
      { \@@_forth_def_put_right:n { \@@_output:n {#1} } }
  }
%    \end{macrocode}
% \end{macro}
%
% \begin{macro}[int]{BASE}
%    \begin{macrocode}
\@@_forth_new_core:nn { BASE }
  { \@@_forth_push:n \l_@@_forth_base_address_int }
%    \end{macrocode}
% \end{macro}
%
% \begin{macro}[int]{BL}
%    \begin{macrocode}
\@@_forth_new_core:nn { BL }
  { \@@_forth_push:n { `\  } }
%    \end{macrocode}
% \end{macro}
%
% \begin{macro}[int]{CHAR}
%   Read a word, leave its first character on the stack.
%    \begin{macrocode}
\@@_forth_new_core:nn { CHAR }
  {
    \@@_forth_input_spaces:
    \@@_forth_input_until:nn { ~ }
      {
        \tl_set:Nx \l_@@_tmpa_tl { \str_head:n {#1} }
        \@@_forth_push:n { \exp_after:wN ` \l_@@_tmpa_tl }
      }
  }
%    \end{macrocode}
% \end{macro}
%
% \begin{macro}[int]{[CHAR]}
%   Read a word, add code to the current definition that leaves its
%   first character on the stack.
%    \begin{macrocode}
\@@_forth_new_compilation_core:nn { [CHAR] }
  {
    \@@_forth_input_spaces:
    \@@_forth_input_until:nn { ~ }
      {
        \tl_set:Nx \l_@@_tmpa_tl { \str_head:n {#1} }
        \@@_forth_def_put_right:x
          { \@@_forth_push:n { \exp_after:wN ` \l_@@_tmpa_tl } }
      }
  }
%    \end{macrocode}
% \end{macro}
%
% \begin{macro}[int]{COUNT}
%   Convert the address of a counted string to its length (on top of the
%   stack), and the address of the first character.  This could be
%   defined with |: COUNT DUP 1+ SWAP @ ;| but this is faster.
%    \begin{macrocode}
\@@_forth_new_core:nn { COUNT }
  {
    \@@_forth_pop_int:N \l_@@_tmpa_int
    \@@_forth_push:nn
      { \l_@@_tmpa_int + 1 }
      { \@@_forth_data_get:N \l_@@_tmpa_int }
  }
%    \end{macrocode}
% \end{macro}
%
% \begin{macro}[int]{CR}
%    \begin{macrocode}
\@@_forth_new_core:nn { CR }
  { \@@_output:x { \iow_newline: } }
%    \end{macrocode}
% \end{macro}
%
% \begin{macro}[int]{DECIMAL}
%    \begin{macrocode}
\@@_forth_new_core:nn { DECIMAL }
  { \@@_forth_data_gset:nn { \l_@@_forth_base_address_int } { 10 } }
%    \end{macrocode}
% \end{macro}
%
% \begin{macro}[int]{EMIT}
%   Output a character with the character code found by popping the data
%   stack.
%    \begin{macrocode}
\@@_forth_new_core:nn { EMIT }
  {
    \@@_forth_pop_int:N \l_@@_tmpa_int
    \@@_output_char:n { \l_@@_tmpa_int }
  }
%    \end{macrocode}
% \end{macro}
%
% ^^A todo: is S" allowed to overlap with the region of WORD or #> ?
% \begin{macro}[int]+S"+ Read a double-quoted string and measure its
%   length \texttt{tmpa}.  Try to allocate that much space for a text
%   literal, decreasing the \texttt{data_text} integer.  If this enters
%   the allotted region, complain, and forget the string.  Otherwise,
%   the string will start at the address \texttt{data_text} (new) and
%   have length \texttt{tmpa}, so we add to the current definition code
%   that pushes those onto the stack.  Finally, store the string, by
%   looping over it (spaces are first protected to avoid losing them),
%   keeping track of the address of each character using \texttt{tmpb}.
%    \begin{macrocode}
\@@_forth_new_compilation_core:nn { S" }
  {
    \@@_forth_input_until:nn "
      { \tl_set:Nn \l_@@_tmpa_tl {#1} }
    \int_set:Nn \l_@@_tmpa_int { \str_count:N \l_@@_tmpa_tl }
    \int_sub:Nn \l_@@_forth_data_text_int { \l_@@_tmpa_int }
    \int_compare:nNnTF
      \l_@@_forth_data_text_int < \l_@@_forth_data_here_int
      {
        \msg_error:nn { runner/forth } { out-of-memory }
        \int_set_eq:NN \l_@@_forth_data_text_int
          \l_@@_forth_data_here_int
      }
      {
        \@@_forth_def_put_right:x
          {
            \@@_forth_push:n
              { \int_use:N \l_@@_forth_data_text_int }
            \@@_forth_push:n
              { \int_use:N \l_@@_tmpa_int }
          }
        \tl_replace_all:Nnn \l_@@_tmpa_tl { ~ } { \ }
        \int_set_eq:NN \l_@@_tmpb_int \l_@@_forth_data_text_int
        \tl_map_inline:Nn \l_@@_tmpa_tl
          {
            \@@_forth_data_gset:nn { \l_@@_tmpb_int } { `#1 }
            \int_incr:N \l_@@_tmpb_int
          }
      }
  }
%    \end{macrocode}
% \end{macro}
%
% \begin{macro}[int]{SPACE}
%    \begin{macrocode}
\@@_forth_new_core:nn { SPACE }
  { \@@_output:n { ~ } }
%    \end{macrocode}
% \end{macro}
%
% \begin{macro}[int]{SPACES}
%    \begin{macrocode}
\@@_forth_new_core:nn { SPACES }
  {
    \@@_forth_pop_int:N \l_@@_tmpa_int
    \@@_output:x { \prg_replicate:nn { \l_@@_tmpa_int } { ~ } }
  }
%    \end{macrocode}
% \end{macro}
%
% \begin{macro}[int]{TYPE}
%    \begin{macrocode}
\@@_forth_new_core:nn { TYPE }
  {
    \@@_forth_pop_int:NN \l_@@_tmpa_int \l_@@_tmpb_int
    \int_step_inline:nnnn
      { \l_@@_tmpa_int }
      { 1 }
      { \l_@@_tmpa_int + \l_@@_tmpb_int - 1 }
      {
        \@@_output_char:n
          { \@@_forth_data_get:n {#1} }
      }
  }
%    \end{macrocode}
% \end{macro}
%
% \begin{macro}[int]{WORD}
%   Pop a character from the stack.  The word will be stored as a
%   counted string in the region from \texttt{data_here} (included) to
%   \texttt{data_word}, so we immediately push \texttt{data_here}.  Then
%   build \cs{l_@@_tmpa_tl} containing the delimiter, and call the
%   appropriate input functions to discard leading delimiters and grab
%   until the next occurrence of the delimiter.  The word found is
%   stored as \cs{l_@@_tmpb_tl}.  Then, after setting
%   \texttt{data_word}, which will index the word as we store it, to
%   \texttt{data_here}, we go through the string, storing its character
%   codes at positions starting from \(1 + \text{\texttt{data_here}}\).
%   Eventually, put the length, obtained by simple subtraction, in the
%   \texttt{data_here} address.  The final increment takes care of
%   placing \texttt{data_word} just after the end of the word.
%    \begin{macrocode}
\group_begin:
  \char_set_catcode_other:n { 0 }
  \@@_forth_new_core:nn { WORD }
    {
      \@@_forth_pop_int:N \l_@@_tmpa_int
      \@@_forth_push:n { \l_@@_forth_data_here_int }
      \group_begin:
      \char_set_lccode:nn { 0 } { \l_@@_tmpa_int }
      \tex_lowercase:D
        {
          \group_end:
          \tl_set:Nn \l_@@_tmpa_tl { ^^@ }
        }
      \exp_args:No \@@_forth_input_discard:N \l_@@_tmpa_tl
      \exp_args:No \@@_forth_input_until:nn \l_@@_tmpa_tl
        { \tl_set:Nn \l_@@_tmpb_tl {#1} }
      \int_set_eq:NN \l_@@_forth_data_word_int
        \l_@@_forth_data_here_int
      \tl_replace_all:Nnn \l_@@_tmpb_tl { ~ } { \  }
      \tl_map_inline:Nn \l_@@_tmpb_tl
        {
          \int_incr:N \l_@@_forth_data_word_int
          \@@_forth_data_gset:nn
            { \l_@@_forth_data_word_int } { `#1 }
        }
      \@@_forth_data_gset:nn
        { \l_@@_forth_data_here_int }
        { \l_@@_forth_data_word_int - \l_@@_forth_data_here_int }
      \int_incr:N \l_@@_forth_data_word_int
      \int_compare:nNnT
        \l_@@_forth_data_word_int > \l_@@_forth_data_text_int
        { \msg_error:nn { runner/forth } { out-of-memory } }
    }
\group_end:
%    \end{macrocode}
% \end{macro}
%
% \subsubsection{Conditional words}
%
% \begin{macro}[int]{IF}
%    \begin{macrocode}
\@@_forth_new_compilation_core:nn { IF }
  {
    \@@_forth_def_put_right_x:n
      { \@@_forth_compiled_if:nw { \if_false: } \fi: }
    \int_incr:N \l_@@_forth_def_nesting_int
  }
%    \end{macrocode}
% \end{macro}
%
% \begin{macro}[int]{ELSE}
%    \begin{macrocode}
\@@_forth_new_compilation_core:nn { ELSE }
  {
    \int_compare:nNnTF \l_@@_forth_def_nesting_int > \c_zero
      {
        \@@_forth_def_put_right_x:n
          { \if_false: { \fi: } { \if_false: } \fi: }
      }
      { \msg_error:nnn { runner/forth } { misplaced } { ELSE } }
  }
%    \end{macrocode}
% \end{macro}
%
% \begin{macro}[int]{THEN}
%    \begin{macrocode}
\@@_forth_new_compilation_core:nn { THEN }
  {
    \int_compare:nNnTF \l_@@_forth_def_nesting_int > \c_zero
      {
        \@@_forth_def_put_right_x:n
          { \if_false: { \fi: } \@@_forth_compiled_then: }
      }
      { \msg_error:nnn { runner/forth } { misplaced } { THEN } }
    \int_decr:N \l_@@_forth_def_nesting_int
  }
%    \end{macrocode}
% \end{macro}
%
% \begin{macro}[int]{\@@_forth_compiled_if:nw, \@@_forth_compiled_then:}
%    \begin{macrocode}
\cs_new_protected:Npn \@@_forth_compiled_if:nw
    #1#2 \@@_forth_compiled_then:
  {
    \@@_forth_pop_int:N \l_@@_tmpa_int
    \int_compare:nNnTF \l_@@_tmpa_int = \c_zero {#2} {#1}
  }
\cs_new_protected_nopar:Npn \@@_forth_compiled_then:
  { \msg_error:nn { runner/forth } { internal } }
%    \end{macrocode}
% \end{macro}
%
% \subsubsection{Looping words}
%
% ^^A todo: Introduce wrapper for "no-interpretation" special core words.
% ^^A todo: change nesting_int to be a stack instead.
% \begin{macro}[int]{DO}
%    \begin{macrocode}
\@@_forth_new_compilation_core:nn { DO }
  {
    \@@_forth_def_put_right_x:n
      { \@@_forth_compiled_do:n { \if_false: } \fi: }
    \int_incr:N \l_@@_forth_def_nesting_int
  }
%    \end{macrocode}
% \end{macro}
%
% \begin{macro}[int]{LOOP}
%    \begin{macrocode}
\@@_forth_new_compilation_core:nn { LOOP }
  {
    \int_compare:nNnTF \l_@@_forth_def_nesting_int > \c_zero
      { \@@_forth_def_put_right_x:n { \if_false: { \fi: } } }
      { \msg_error:nnn { runner/forth } { misplaced } { LOOP } }
    \int_decr:N \l_@@_forth_def_nesting_int
  }
%    \end{macrocode}
% \end{macro}
%
% ^^A todo: optimize
% \begin{macro}[int]{\@@_forth_compiled_do:n}
%    \begin{macrocode}
\cs_new_protected:Npn \@@_forth_compiled_do:n #1
  {
    \@@_forth_pop_int:NN \l_@@_tmpa_int \l_@@_tmpb_int
    \@@_forth_return_push:n { \l_@@_tmpa_int } % ^^A limit
    \@@_forth_return_push:n { \l_@@_tmpb_int } % ^^A index
    \int_do_until:nNnn \l_@@_tmpb_int = \l_@@_tmpa_int
      {
        #1
        \@@_forth_return_pop_int:N \l_@@_tmpb_int
        \int_incr:N \l_@@_tmpb_int
        \@@_forth_return_pop_int:N \l_@@_tmpa_int
        \@@_forth_return_push:n { \l_@@_tmpa_int }
        \@@_forth_return_push:n { \l_@@_tmpb_int }
      }
    \@@_break_point:n
      {
        \@@_forth_return_pop_int:N \l_@@_tmpb_int
        \@@_forth_return_pop_int:N \l_@@_tmpa_int
      }
  }
%    \end{macrocode}
% \end{macro}
%
% \begin{macro}[int]{I, J}
%    \begin{macrocode}
\@@_forth_new_compilation_core:nn { I }
  {
    \@@_forth_def_put_right:n
      {
        \@@_forth_push:n
          {
            \@@_array_item:Nn \g_@@_forth_array
              { \l_@@_forth_return_here_int - 1 }
          }
      }
  }
\@@_forth_new_compilation_core:nn { J }
  {
    \@@_forth_def_put_right_x:n
      {
        \@@_forth_push:n
          {
            \@@_array_item:Nn \g_@@_forth_array
              { \l_@@_forth_return_here_int - 3 }
          }
      }
  }
%    \end{macrocode}
% \end{macro}
%
% \begin{macro}[int]{LEAVE}
%   Break the current loop, which takes care of removing the loop
%   control parameters from the return stack.
%    \begin{macrocode}
\@@_forth_new_compilation_core:nn { LEAVE }
  { \@@_forth_def_put_right:n { \@@_break:n { } } }
%    \end{macrocode}
% \end{macro}
%
% \begin{macro}[int]{UNLOOP}
%   Drop the loop control parameters from the return stack.
%    \begin{macrocode}
\@@_forth_new_compilation_core:nn { UNLOOP }
  {
    \@@_forth_return_pop_int:N \l_@@_tmpa_int
    \@@_forth_return_pop_int:N \l_@@_tmpb_int
  }
%    \end{macrocode}
% \end{macro}
%
% \subsubsection{Ending words}
%
% When there is no more input, the interpreter sees an empty word.  Its
% execution semantics is to stop the interpreter by setting a boolean.
% This also works to stop when in compilation mode.
%    \begin{macrocode}
\@@_forth_new_immediate_core:nn { }
  { \bool_set_true:N \l_@@_forth_stop_bool }
%    \end{macrocode}
%
% \subsubsection{Environmental queries}
%
% \begin{macro}[int]{ENVIRONMENT?}
% \begin{macro}[aux]{\@@_forth_environment_aux:n}
%   Pop an address \(a\) and a length \(b\) from the stack, and extract
%   a comma-list of the values from \(a\) to \(a + b - 1\) inclusive
%   (the query string).  If that string matches any known one, return
%   the result and a true flag, otherwise return a false flag.
%    \begin{macrocode}
\@@_forth_new_core:nn { ENVIRONMENT? }
  {
    \@@_forth_pop_int:NN \l_@@_tmpa_int \l_@@_tmpb_int
    \prop_get:NxNTF \g_@@_forth_environment_prop
      {
        \int_step_function:nnnN
          { \l_@@_tmpa_int }
          { 1 }
          { \l_@@_tmpa_int + \l_@@_tmpb_int - 1 }
          \@@_forth_environment_aux:n
      }
      \l_@@_tmpa_tl
      {
        \l_@@_tmpa_tl
        \@@_forth_push_true:
      }
      {
        \@@_forth_push_false:
      }
  }
\cs_new:Npn \@@_forth_environment_aux:n #1
  { \@@_forth_data_get:n {#1} , }
%    \end{macrocode}
% \end{macro}
% \end{macro}
%
% \begin{variable}{\g_@@_forth_environment_prop}
% \begin{macro}{\@@_forth_environment_def:nn}
%   Keys are known string, in the form of comma lists of their character
%   codes.  Currently, the following exist:
%   \begin{itemize}
%   \item ADDRESS-UNIT-BITS: 24
%   \item CORE: false
%   \item CORE-EXT: false
%   \item FLOORED: false
%   \item MAX-D: 8388607 and 16777215
%   \item MAX-N: 8388607
%   \item MAX-U: 16777215
%   \item MAX-UD: 16777215 and 16777215.
%   \end{itemize}
%   Values are the code that pushes the appropriate values to the stack.
%    \begin{macrocode}
\prop_new:N \g_@@_forth_environment_prop
\group_begin:
  \cs_set_protected:Npn \@@_forth_environment_def:nn #1#2
    {
      \prop_gput:Nxn \g_@@_forth_environment_prop
        { \tl_map_function:nN {#1} \@@_tmp:w }
        {#2}
    }
  \cs_set:Npn \@@_tmp:w #1 { \int_eval:n { `#1 } , }
  \@@_forth_environment_def:nn { ADDRESS-UNIT-BITS }
    { \@@_forth_push:n { 24 } }
  \@@_forth_environment_def:nn { CORE }
    { \@@_forth_push_false: }
  \@@_forth_environment_def:nn { CORE-EXT }
    { \@@_forth_push_false: }
  \@@_forth_environment_def:nn { FLOORED }
    { \@@_forth_push_false: }
  \@@_forth_environment_def:nn { MAX-D }
    {
      \@@_forth_push:n { 16777215 }
      \@@_forth_push:n { 8388607 }
    }
  \@@_forth_environment_def:nn { MAX-N }
    { \@@_forth_push:n { 8388607 } }
  \@@_forth_environment_def:nn { MAX-U }
    { \@@_forth_push:n { 16777215 } }
  \@@_forth_environment_def:nn { MAX-UD }
    {
      \@@_forth_push:n { 16777215 }
      \@@_forth_push:n { 16777215 }
    }
  \@@_forth_environment_def:nn { /PAD }
    {
      \@@_forth_push:n
        { \g_@@_forth_pad_max_int - \g_@@_forth_pad_min_int }
    }
  \@@_forth_environment_def:nn { RETURN-STACK-CELLS }
    {
      \@@_forth_push:n
        { \g_@@_forth_return_max_int - \g_@@_forth_return_min_int }
    }
  \@@_forth_environment_def:nn { STACK-CELLS }
    {
      \@@_forth_push:n
        { \g_@@_forth_stack_max_int - \g_@@_forth_stack_min_int }
    }
\group_end:
%    \end{macrocode}
% \end{macro}
% \end{variable}
%
% \subsubsection{Misc words}
%
% \begin{macro}[int]{(}%^^A)
%   Comments are implemented by grabbing a piece of input delimited by a
%   right parenthesis, and discarding it with \cs{use_none:n}.  This is
%   an immediate word.
%    \begin{macrocode}
\@@_forth_new_immediate_core:nn (
  { \@@_forth_input_until:nN ) \use_none:n }
%    \end{macrocode}
% \end{macro}
%
% \subsection{Running the interpreter}
%
% \begin{macro}[int]{\@@_forth_run:}
%   The interpreter is an infinite loop retrieving a word at each
%   iteration.  When there is nothing left, the word fed to
%   \cs{@@_forth_interpret:n} is empty, and this word is defined to set
%   the boolean \cs{l_@@_forth_stop_bool}, so that the loop stops.
%    \begin{macrocode}
\cs_new_protected_nopar:Npn \@@_forth_run:
  {
    \@@_forth_init:
    \bool_set_false:N \l_@@_forth_stop_bool
    \bool_until_do:Nn \l_@@_forth_stop_bool
      {
        \@@_forth_input_spaces:
        \@@_forth_input_until:nN { ~ } \@@_forth_interpret:n
      }
  }
%    \end{macrocode}
% \end{macro}
%
% ^^A todo: use proper alloc routine for the base.
% \begin{macro}[int]{\@@_forth_init:}
% \begin{macro}[aux, EXP]{\@@_forth_cleanup:n}
%   Before starting the interpreter loop, we concatenate the program
%   string and the input string, with a space in between, then replace
%   all non-graphical characters by spaces (some care is needed to avoid
%   losing spaces), then set some of the initial values.
%    \begin{macrocode}
\cs_new_protected_nopar:Npn \@@_forth_init:
  {
    \@@_forth_init_data:
    \tl_use:N \g_@@_forth_init_toks_tl
    \int_set_eq:NN
      \l_@@_forth_toks_int
      \g_@@_forth_core_toks_int
    \prop_set_eq:NN
      \l_@@_forth_words_prop
      \g_@@_forth_core_words_prop
    \int_set_eq:NN \l_@@_forth_base_address_int
      \l_@@_forth_data_here_int
    \@@_forth_put_here:n { 10 }
    \tl_set:Nx \l_@@_tmpa_tl
      { \l_@@_program_str \c_space_tl \l_@@_input_str }
    \tl_replace_all:Nnn \l_@@_tmpa_tl { ~ } { { ~ } }
    \str_set:Nx \l_@@_forth_input_str
      {
        \tl_map_function:NN \l_@@_tmpa_tl
          \@@_forth_cleanup:n
      }
  }
\cs_new:Npn \@@_forth_cleanup:n #1
  { \int_compare:nTF { 31 < `#1 < 128 } {#1} { ~ } }
%    \end{macrocode}
% \end{macro}
% \end{macro}
%
% \begin{macro}[int]{\@@_forth_interpret:n}
%   Use the interpretation semantics of |#1|.  If this fails, we hope to
%   have a number, which should be parsed as such.
%    \begin{macrocode}
\cs_new_protected:Npn \@@_forth_interpret:n #1
  {
    \prop_get:NnNTF \l_@@_forth_words_prop {#1} \l_@@_tmpa_tl
      {
        \tex_the:D \tex_toks:D
          \exp_after:wN \use_none:n \l_@@_tmpa_tl \scan_stop:
      }
      {
        \@@_forth_get_number:nNTF {#1} \l_@@_tmpa_tl
          { \@@_forth_push:n { \l_@@_tmpa_tl } }
          { \msg_error:nnn { runner/forth } { unknown-word } {#1} }
      }
  }
%    \end{macrocode}
% \end{macro}
%
% \subsection{Messages}
%
% \begin{macro}[int, EXP]{\@@_forth_footprint:}
%    \begin{macrocode}
\cs_new_nopar:Npn \@@_forth_footprint:
  {
    Summary~of~memory~use.\\\\
    Input~buffer: \
      \int_use:N \g_@@_forth_input_min_int \ to \
      \int_use:N \g_@@_forth_input_max_int . \\
    Pad: \
      \int_use:N \g_@@_forth_pad_min_int \ to \
      \int_use:N \g_@@_forth_pad_max_int .\\
    Data:
      \int_use:N \g_@@_forth_data_min_int \ to \
      \int_use:N \g_@@_forth_data_max_int , \ split\ as \\
    \iow_indent:n
      {
        \int_eval:n
          { \l_@@_forth_data_here_int - \g_@@_forth_data_min_int }
        \ cells\ of\ variables\ and\ allotted\ data, \\
        \int_compare:nTF
          {
            \l_@@_forth_data_here_int
            <= \l_@@_forth_data_word_int
            <= \l_@@_forth_data_text_int
          }
          {
            \int_eval:n
              {
                \l_@@_forth_data_word_int
                - \l_@@_forth_data_text_int
              }
            \ cells\ for\ WORD\ or\ \#>\ .\\
            \int_eval:n
              {
                \l_@@_forth_data_text_int
                - \l_@@_forth_data_word_int
              }
            \ unassigned\ cells. \\
          }
          {
            \int_eval:n
              {
                \l_@@_forth_data_text_int
                - \l_@@_forth_data_here_int
              }
            \ unassigned\ cells. \\
          }
        \int_eval:n
          { \g_@@_forth_data_max_int - \l_@@_forth_data_text_int }
        \ cells\ for\ literal\ text. \\
      }
    Stack: \
      \int_use:N \g_@@_forth_stack_min_int \ to \
      \int_use:N \g_@@_forth_stack_max_int , \
      \int_eval:n
        { \l_@@_forth_stack_here_int - \g_@@_forth_stack_min_int }
      \ out\ of\
      \int_eval:n
        { \g_@@_forth_stack_max_int - \g_@@_forth_stack_min_int }
      \ occupied.
    \\
    Return stack: \
      \int_use:N \g_@@_forth_return_min_int \ to \
      \int_use:N \g_@@_forth_return_max_int , \
      \int_eval:n
        { \l_@@_forth_return_here_int - \g_@@_forth_return_min_int }
      \ out\ of\
      \int_eval:n
        { \g_@@_forth_return_max_int - \g_@@_forth_return_min_int }
      \ occupied.
    \\
  }
%    \end{macrocode}
% \end{macro}
%
%    \begin{macrocode}
\msg_new:nnn { runner/forth } { unknown-word }
  { The~word~'#1'~is~not~defined. }
\msg_new:nnn { runner/forth } { empty-stack }
  { The~data~stack~is~empty,~and~there~is~nothing~to~retrieve~there. }
\msg_new:nnn { runner/forth } { empty-return-stack }
  { The~return~stack~is~empty,~and~there~is~nothing~to~retrieve~there. }
\msg_new:nnnn { runner/forth } { out-of-memory }
  { The~Forth~interpreter~ran~out~of~memory. }
  { \@@_forth_footprint: }
\msg_new:nnn { runner/forth } { out-of-bounds }
  { ALLOT~was~called~with~a~negative~argument~that~made~HERE~=~#1. }
\msg_new:nnn { runner/forth } { no-interpretation }
  { The~word~#1~can~only~be~used~in~definitions. }
\msg_new:nnn { runner/forth } { no-def-immediate }
  { Somehow~the~word~#1~cannot~be~found~in~the~dictionary }
\msg_new:nnn { runner/forth } { too-many-ifs }
  { More~IFs~than~THENs~in~this~definition! }
\msg_new:nnn { runner/forth } { misplaced }
  { Misplaced~#1. }
\msg_new:nnn { runner/forth } { internal }
  { Internal~error.~Please~report. }
%    \end{macrocode}
%
% \end{runner-implementation}
% \endinput